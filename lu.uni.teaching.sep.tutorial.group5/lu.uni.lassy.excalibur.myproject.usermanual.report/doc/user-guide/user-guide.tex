\chapter{Usage Guide}
\label{chap:usage_guide}

This section is aimed at describing the general use of the software, since it is
\textbf{deployed, configured} and \textbf{run}.

This software is used by actors. These actors rely on the software to perform a
set of business activities (called here procedures) aimed at reaching a
particular goal. 

These prodedures are splet in two groups:
\begin{itemize}
  \item \textbf{Multi-procedures:} which are procedures at \textbf{summary} or
  \textbf{user-goal} level involving several active or pro-active actors.
  Each of these procedures aims at illustrating intertwined
  business activities required to be performed by the involved actors
  to reach the expected goal. Each business activity between the system and an
  actor must correspond to a \textbf{system operation} instance given with actual parameter values.

  \item \textbf{Mono-procedures:} which are procedures at \textbf{summary} or
  \textbf{user-goal} level involving only one active or pro-active actor.
  Each of these procedures aims at illustrating the required business
  activities an actor has to perform to reach the expected goal. Each business
  activity between the system and the actor must correspond to a \textbf{system
  operation} instance given with actual parameter values.

\end{itemize}




Each process has to be documented using the following textual description
template \cite{armour01usecase} \textbf{BUT its content must be as low level as possible with actual values}:
\vspace{0.5cm}
\hrule
\begin{lyxlist}{PC1}
\small{
\item [\textbf{Procedure:}] ProcessMissionOne
\item [\textbf{Scope:}] Crisis Management System (\emph{CMS})
\item [\textbf{Primary Actor}:] Coordinator John
\item [\textbf{Secondary Actor(s)}:] FirstAidWorker Bob,\\
                  ExternalResourceSystem ERS
\item [\textbf{Goal:}] The intention of the Coordinator is to process mission
with ID equal to 1.
\item [\textbf{Level}:] User-goal level
\item [\textbf{Main~Success~Scenario}]:\\
1. \emph{John} instructs the \emph{CMS} to process the mission with ID equal to 12.031005\\
2. \emph{CMS} selects the internal worker \emph{Bob} to execute the mission 12.031005\\
3. \emph{CMS} instructs \emph{Bob} to behave as \emph{First Aid Worker (FAW)}\\
4. \emph{Bob} informs the \emph{CMS} of his arrival\\
5. \emph{Bob} informs the \emph{CMS} that he starts to execute the mission 12.031005\\
6. \emph{Bob} informs the \emph{CMS} that the mission 12.031005 outcome is ``Mission completed''


\item [\textbf{Extensions}]:\\
2.a None internal worker can execute the mission\\
\hspace*{0.5cm} 2.a.1 \emph{CMS} sends a request for an external resource to the \emph{ERS} actor instance\\
\hspace*{0.5cm} 2.a.2 \emph{ERS} informs \emph{CMS} that the request can be processed\\
\hspace*{0.5cm} 2.a.3 \emph{ERS} informs \emph{CMS} that \emph{Bob} can now be selected as first aid worker\\
\hspace*{0.5cm} \textbf{procedure continues at step 3}

}

\end{lyxlist}
\hrule
\vspace{0.5cm}




\Remark{Processes presentation}: processes should be introduced to the
reader in a pedagogical manner. Thus, simple and common processes should be presented before
than more complex and less utilised ones.

\Remark{Graphical User Interfaces (GUIs)}: include GUIs screenshots to show the
different stages of the process while its is performed by the actor(s).


\section{Librarians}
\label{chap:usage_guide}

\subsection{Multi-procedures}

\vspace{0.5cm}
\hrule
\begin{lyxlist}{PC1}
\small{
\item [\textbf{Procedure:}] Librarian login
\item [\textbf{Scope:}] LibrarianSystem (\emph{LibSys})
\item [\textbf{Primary Actor}:] Librarian Smith
\item [\textbf{Secondary Actor(s)}:] LoginSystem LS
\item [\textbf{Goal:}] The intention of the User is to login into the
system in order to use the services of the application.
\item [\textbf{Level}:] User-goal level
\item [\textbf{Main~Success~Scenario}]:\\
1. \emph{Smith} enters the user name and password in the respective fields\\
2. \emph{Smith} presses on the login button which sends a login request message
to \emph{LS}\\
3. \emph{LS} logs \emph{Smith} into the application\\


\item [\textbf{Extensions}]:\\
3.a Username of \emph{Smith} not found\\
\hspace*{0.5cm} 3.a.1 \emph{LS} returns a message stating that the user name
or password is not correct \\
\hspace*{0.5cm} \textbf{procedure returns to step 1}

3.b Password of \emph{Smith} not correct \\
\hspace*{0.5cm} 3.b.1 \emph{LS} returns a message stating that the user name
or password is not correct \\
\hspace*{0.5cm} \textbf{procedure returns to step 1}

}

\end{lyxlist}
\hrule

\vspace{0.5cm}
\hrule
\begin{lyxlist}{PC1}
\small{
\item [\textbf{Procedure:}] Grant book loan
\item [\textbf{Scope:}] LibrarianSystem (\emph{LibSys})
\item [\textbf{Primary Actor}:] Librarian Smith
\item [\textbf{Secondary Actor(s)}:] Customer John
\item [\textbf{Goal:}] The intention of the User is to grant a loan for a book
requested by a customer.
\item [\textbf{Level}:] User-goal level
\item [\textbf{Main~Success~Scenario}]:\\
1. \emph{John} is approaching the librarian with a book he wants to loan. \\
2. \emph{John} gives the librarian his ID so the librarian can look up his
account\\
3. \emph{Smith} opens up \emph{John}'s account and clicks on ''Add loan''. \\
4. \emph{Smith} enters the book ID in the popup window and clicks on ''Add
Loan''.\\

\item [\textbf{Extensions}]:\\
2.a The librarian cannot verify the customer account\\
3.a \emph{Smith} decides not to allow the loan, either due to too many active
loans, or because a suspended or banned account.\\
\hspace*{0.5cm} 3.a.1 \emph{John} gets told by \emph{Smith} that he cannot
currently loan a book and why.\\ 
\hspace*{0.5cm} \textbf{procedure \emph{ends}}

}

\end{lyxlist}
\hrule

%

\vspace{0.5cm}
\hrule
\begin{lyxlist}{PC1}
\small{
\item [\textbf{Procedure:}] Close book loan
\item [\textbf{Scope:}] LibrarianSystem (\emph{LibSys})
\item [\textbf{Primary Actor}:] Librarian Smith
\item [\textbf{Secondary Actor(s)}:] Customer John
\item [\textbf{Goal:}] The intention of the User is to close a loan for a book
returned by a customer.
\item [\textbf{Level}:] User-goal level
\item [\textbf{Main~Success~Scenario}]:\\
1. \emph{John} is approaching the librarian with a book he wants to return. \\
2. \emph{John} gives the librarian his ID so the librarian can look up his
account\\
3. \emph{Smith} opens up \emph{John}'s account and selects the book \emph{John}
wants to return.\\
4. \emph{Smith} then clicks on ''Return Book''.\\

\item [\textbf{Extensions}]:\\
2.a The librarian cannot verify the customer account\\
3.a \emph{Smith} sees that either the book \emph{John} wants to return is
overdue and John needs to pay a fine, or due to too many overdues his account
has been suspended/banned.\\
\hspace*{0.5cm} 3.a.1 \emph{John} gets told by \emph{Smith} that he needs to
pay fine since the books is overdue or his account is suspended/banned.\\
\hspace*{0.5cm} \textbf{procedure \emph{ends}}

}

\end{lyxlist}
\hrule

%%
%%

\vspace{0.5cm}
\hrule
\begin{lyxlist}{PC1}
\small{
\item [\textbf{Procedure:}] Suspend Customer
\item [\textbf{Scope:}] LibrarianSystem (\emph{LibSys})
\item [\textbf{Primary Actor}:] Librarian Smith
\item [\textbf{Secondary Actor(s)}:] Customer John
\item [\textbf{Goal:}] The intention of the User is to suspend the customer
account until a certain fine is paid.
\item [\textbf{Level}:] User-goal level
\item [\textbf{Main~Success~Scenario}]:\\
1. \emph{Smith} has been made aware of a customer with multiple
overdue books.\\
2. \emph{Smith} opens this customers account and clicks on ''Suspend
Customer''\\
3. \emph{Smith} enters a small descriptive note in the text field on the
customer page.\\
4. \emph{John} now can't loan any more books until the fine is paid off and
every librarian accessign his account sees what the fine is.\\

}

\end{lyxlist}
\hrule

\vspace{0.5cm}
\hrule
\begin{lyxlist}{PC1}
\small{
\item [\textbf{Procedure:}] Reinstate Customer
\item [\textbf{Scope:}] LibrarianSystem (\emph{LibSys})
\item [\textbf{Primary Actor}:] Librarian Smith
\item [\textbf{Secondary Actor(s)}:] Customer John
\item [\textbf{Goal:}] The intention of the User is to reinstate a suspended
customer account.
\item [\textbf{Level}:] User-goal level
\item [\textbf{Main~Success~Scenario}]:\\
1. \emph{John} has been made aware that his customer account got suspended
and he needs to pay a fine to reinstate it.\\
2. \emph{John} goes to the library and gives \emph{Smith} his customer ID so
he can pull up his account.\\
3. \emph{Smith} opens this customers account and checks the fine \emph{John}
needs to pay.\\
4. \emph{Jonh} pays \emph{Smith} the fine.\\
5. \emph{Smith} receives the payment and clicks on ''Reinstate Account''.\\

\item [\textbf{Extensions}]:\\
4.a \emph{John} doesn't have enough money on him to pay the fine and thus
cannot complete the reinstation of his customer account.\\
}

\end{lyxlist}
\hrule

\vspace{0.5cm}
\hrule
\begin{lyxlist}{PC1}
\small{
\item [\textbf{Procedure:}] Ban Customer
\item [\textbf{Scope:}] LibrarianSystem (\emph{LibSys})
\item [\textbf{Primary Actor}:] Librarian Smith
\item [\textbf{Secondary Actor(s)}:] LibrarianSupervisor Jane, SysAdmin Fil
\item [\textbf{Goal:}] The intention of the User is to put in a request to ban a
customer.
\item [\textbf{Level}:] User-goal level
\item [\textbf{Main~Success~Scenario}]:\\
1. \emph{Smith} has been made aware of a deviant customer with multiple
infractions.\\
2. \emph{Smith} opens this customers account and click on ''Ban Customer''\\
3. \emph{Smith} enters a small descriptive text and clicks ''Send''.\\
4. \emph{Jane} receives this request and approves this.\\
5. \emph{Fil} sees the request has been approved and closes the customer
account.\\

\item [\textbf{Extensions}]:\\
4.a \emph{Jane} denies the request to close the account and thus \emph{Fil}
never gets involved.\\
}

\end{lyxlist}
\hrule

\vspace{0.5cm}
\hrule
\begin{lyxlist}{PC1}
\small{
\item [\textbf{Procedure:}] Unban Customer
\item [\textbf{Scope:}] LibrarianSystem (\emph{LibSys})
\item [\textbf{Primary Actor}:] Librarian Smith
\item [\textbf{Secondary Actor(s)}:] LibrarianSupervisor Jane, SysAdmin Fil,
Customer John
\item [\textbf{Goal:}] The intention of the User is to put in a request to unban
a customer.
\item [\textbf{Level}:] User-goal level
\item [\textbf{Main~Success~Scenario}]:\\
1. \emph{John} has paid off all his fines at the library and his ban has been
in effect for quite some time and he wants to be unbanned and promises to
follow the rules.\\
2. \emph{Smith} opens this customers account and click on ''Unban Customer''\\
3. \emph{Smith} enters a small descriptive text and clicks ''Send''.\\
4. \emph{Jane} receives this request and approves this.\\
5. \emph{Fil} sees the request has been approved and reopens the customer
account.\\

\item [\textbf{Extensions}]:\\
1.a \emph{Smith} sees that \emph{John} actually hasn't paid of all his fines or
for whatever reason doesn't think \emph{John} shouldn't be allowed to put in a
request to reopen his account.\\
4.a \emph{Jane} denies the request to reopen the account and thus \emph{Fil}
never gets involved.\\
}

\end{lyxlist}
\hrule

%%%%%%%%%%

\vspace{0.5cm}
\hrule
\begin{lyxlist}{PC1}
\small{
\item [\textbf{Procedure:}] Send message.
\item [\textbf{Scope:}] LibrarianSystem (\emph{LibSys})
\item [\textbf{Primary Actor}:] Librarian Smith
\item [\textbf{Secondary Actor(s)}:] SysAdmin Fil
\item [\textbf{Goal:}] The intention of the User is to send a message to the
system administrator.
\item [\textbf{Level}:] User-goal level
\item [\textbf{Main~Success~Scenario}]:\\
1. \emph{Smith} is in any database or the homescreen.\\
2. \emph{Smith} clicks the button ''Messages''.\\
3. \emph{Smith} moves his cursor into the text field below the messages.\\
4. \emph{Smith} enters his message for \emph{Fil} into the field.\\
5. \emph{Smith} now clicks on ''Send''.

}

\end{lyxlist}
\hrule

\subsection{Mono-procedures}

\vspace{0.5cm}
\hrule
\begin{lyxlist}{PC1}
\small{
\item [\textbf{Procedure:}] Search books list
\item [\textbf{Scope:}] LibrarianSystem (\emph{LibSys})
\item [\textbf{Primary Actor}:] Librarian Smith
\item [\textbf{Secondary Actor(s)}:] 
\item [\textbf{Goal:}] The intention of the User is to search the book database
for a specific book.
\item [\textbf{Level}:] User-goal level
\item [\textbf{Main~Success~Scenario}]:\\
1. \emph{Smith} is at the home panel of the application after logging in.\\
2. \emph{Smith} now clicks on the button ''Books'' to get to the book database
and search for books.\\
3. In the search field on top \emph{Smith} enters either the book name, ISBN or
author of the book he wants to find.\\
4. In the dropdown menu next to the search field \emph{Smith} chooses which
property he just entered.\\
5. \emph{Smith} presses ''Search'' to get a list of all the books matching his
search query.\\


\item [\textbf{Extensions}]:\\
5.a \emph{Smith} left the search field blank, which then prints him a list of
all the books in the book database.\\
}

%%

\end{lyxlist}
\hrule

\vspace{0.5cm}
\hrule
\begin{lyxlist}{PC1}
\small{
\item [\textbf{Procedure:}] Search customer list
\item [\textbf{Scope:}] LibrarianSystem (\emph{LibSys})
\item [\textbf{Primary Actor}:] Librarian Smith
\item [\textbf{Secondary Actor(s)}:] 
\item [\textbf{Goal:}] The intention of the User is to search the customer
database for a specific customer.
\item [\textbf{Level}:] User-goal level
\item [\textbf{Main~Success~Scenario}]:\\
1. \emph{Smith} is at the home panel of the application after logging in.\\
2. \emph{Smith} now clicks on the button ''Customers'' to get to the customer
database and search for customers.\\
3. In the search field on top \emph{Smith} enters either the first name, last
name, customer ID or account status of the customer he wants to find.\\
4. In the dropdown menu next to the search field \emph{Smith} chooses which
property he just entered.\\
5. \emph{Smith} presses ''Search'' to get a list of all the customers matching
his search query.\\


\item [\textbf{Extensions}]:\\
5.a \emph{Smith} left the search field blank, which then prints him a list of
all the customers in the customer database.\\
}

\end{lyxlist}
\hrule

%%

\vspace{0.5cm}
\hrule
\begin{lyxlist}{PC1}
\small{
\item [\textbf{Procedure:}] Inspect customer
\item [\textbf{Scope:}] LibrarianSystem (\emph{LibSys})
\item [\textbf{Primary Actor}:] Librarian Smith
\item [\textbf{Secondary Actor(s)}:] 
\item [\textbf{Goal:}] The intention of the User is to inspect a customers
profile to get further information about current or past loans, comments from
other librarians, or take further actions with his account.
\item [\textbf{Level}:] User-goal level
\item [\textbf{Main~Success~Scenario}]:\\
1. \emph{Smith} successfully launched a search query on the customer database.\\
2. \emph{Smith} now selects the customer he wishes to inspect from the list
displayed in the application matching his search query.\\
3. A new window will popup displaying per default current loans, with the
possibility to show past loans instead, as well as giving the librarian
the possibility to grant a loan, make a book return, suspend or reinstate the
account, request a ban or unban and add notes to his page.\\


\item [\textbf{Extensions}]:\\
}

\end{lyxlist}
\hrule

%%
%%%%%%%%%%%%%%%%%%%%%%%%

\vspace{0.5cm}
\hrule
\begin{lyxlist}{PC1}
\small{
\item [\textbf{Procedure:}] Show loan history
\item [\textbf{Scope:}] LibrarianSystem (\emph{LibSys})
\item [\textbf{Primary Actor}:] Librarian Smith
\item [\textbf{Secondary Actor(s)}:] 
\item [\textbf{Goal:}] The intention of the User is to check a customers loan
history
\item [\textbf{Level}:] User-goal level
\item [\textbf{Main~Success~Scenario}]:\\
1. \emph{Smith} is at the account page of a customer.\\
2. \emph{Smith} clicks the button on the bottom right titled ''History'' which
replaces the current table of current loans which a table of past loans as well
as replacing the button with another one to get back to current loans.


\item [\textbf{Extensions}]:\\
}

\end{lyxlist}
\hrule


%%%%%%%%%%%%%%%%%%%%%%

\vspace{0.5cm}
\hrule
\begin{lyxlist}{PC1}
\small{
\item [\textbf{Procedure:}] Edit customer notes.
\item [\textbf{Scope:}] LibrarianSystem (\emph{LibSys})
\item [\textbf{Primary Actor}:] Librarian Smith
\item [\textbf{Secondary Actor(s)}:] 
\item [\textbf{Goal:}] The intention of the User is to edit the notes and
comments on a customers account page.
\item [\textbf{Level}:] User-goal level
\item [\textbf{Main~Success~Scenario}]:\\
1. \emph{Smith} is at the account page of a customer.\\
2. \emph{Smith} clicks in the text box on the bottom of the window and adds his
notes and comments on the customers in this field like fines to be paid or
suspension issued at date DD/MM/YYYY.


\item [\textbf{Extensions}]:\\
}

\end{lyxlist}
\hrule

%%%%%%%%%%%%%%%%%%%

\vspace{0.5cm}
\hrule
\begin{lyxlist}{PC1}
\small{
\item [\textbf{Procedure:}] Add book.
\item [\textbf{Scope:}] LibrarianSystem (\emph{LibSys})
\item [\textbf{Primary Actor}:] Librarian Smith
\item [\textbf{Secondary Actor(s)}:] 
\item [\textbf{Goal:}] The intention of the User is to add a book to the
book database.
\item [\textbf{Level}:] User-goal level
\item [\textbf{Main~Success~Scenario}]:\\
1. \emph{Smith} is at the book database menu.\\
2. \emph{Smith} clicks on the button labeled ''Add Book''.\\
3. \emph{Smith} enters the books ISBN into the popup window and clicks on ''Add
Book''.\\
4. The system now automatically adds a new entry for this book, looking up the
author and title associated with the ISBN and generating a new book ID for this
book.\\


\item [\textbf{Extensions}]:\\
3.a \emph{Smith} entered a wrong ISBN prompting him to either cancel the
procedure or manually adding book name and author name.\\
}

\end{lyxlist}
\hrule


%%%%%%%%%%

\vspace{0.5cm}
\hrule
\begin{lyxlist}{PC1}
\small{
\item [\textbf{Procedure:}] Remove book.
\item [\textbf{Scope:}] LibrarianSystem (\emph{LibSys})
\item [\textbf{Primary Actor}:] Librarian Smith
\item [\textbf{Secondary Actor(s)}:] 
\item [\textbf{Goal:}] The intention of the User is to remove a book from the
book database.
\item [\textbf{Level}:] User-goal level
\item [\textbf{Main~Success~Scenario}]:\\
1. \emph{Smith} is at the book database menu.\\
2. \emph{Smith} selects the book he wishes to remove from the book database.\\
3. \emph{Smith} clicks on the button labeled ''Remove Book''.\\
4. The system now removes this particular copy of a book from the
database.\\


\item [\textbf{Extensions}]:\\
3.a \emph{Smith} chose a book currently being loaned which will bring up a
prompt telling him he can't remove this copy from the database.\\
3.b \emph{Smith} chose a book currently being reserved by a customer which will
bring up a prompt if he really wants to remove this book from the database as
this will cancel the reservation.\\
}

\end{lyxlist}
\hrule

%%%%%%%%%%

\vspace{0.5cm}
\hrule
\begin{lyxlist}{PC1}
\small{
\item [\textbf{Procedure:}] Check Loan\&Reservation requests.
\item [\textbf{Scope:}] LibrarianSystem (\emph{LibSys})
\item [\textbf{Primary Actor}:] Librarian Smith
\item [\textbf{Secondary Actor(s)}:] 
\item [\textbf{Goal:}] The intention of the User is to check loan and
reservation requests made via the mobile app.
\item [\textbf{Level}:] User-goal level
\item [\textbf{Main~Success~Scenario}]:\\
1. \emph{Smith} is in any database or the homescreen.\\
2. \emph{Smith} clicks the button ''Loan\&Reservation requests''.\\
3. \emph{Smith} now sees all loan and reservation requests made via the
mobile app and decide whether or not to grant them.\\

}

\end{lyxlist}
\hrule

%%%%%%%%%%

\vspace{0.5cm}
\hrule
\begin{lyxlist}{PC1}
\small{
\item [\textbf{Procedure:}] Grant Loan\&Reservation requests.
\item [\textbf{Scope:}] LibrarianSystem (\emph{LibSys})
\item [\textbf{Primary Actor}:] Librarian Smith
\item [\textbf{Secondary Actor(s)}:] 
\item [\textbf{Goal:}] The intention of the User is to grant a loan or
reservation requests made via the mobile app.
\item [\textbf{Level}:] User-goal level
\item [\textbf{Main~Success~Scenario}]:\\
1. \emph{Smith} is in any database or the homescreen.\\
2. \emph{Smith} clicks the button ''Loan\&Reservation requests''.\\
3. \emph{Smith} selects one of the requests.\\
4. \emph{Smith} now presses the button below marked ''Grant
Loan/Reservation''.\\

}

\end{lyxlist}
\hrule

%%%%%%%%%%

\vspace{0.5cm}
\hrule
\begin{lyxlist}{PC1}
\small{
\item [\textbf{Procedure:}] Deny Loan\&Reservation requests.
\item [\textbf{Scope:}] LibrarianSystem (\emph{LibSys})
\item [\textbf{Primary Actor}:] Librarian Smith
\item [\textbf{Secondary Actor(s)}:] 
\item [\textbf{Goal:}] The intention of the User is to deny a loan or
reservation requests made via the mobile app.
\item [\textbf{Level}:] User-goal level
\item [\textbf{Main~Success~Scenario}]:\\
1. \emph{Smith} is in any database or the homescreen.\\
2. \emph{Smith} clicks the button ''Loan\&Reservation requests''.\\
3. \emph{Smith} selects one of the requests.\\
4. \emph{Smith} now presses the button below marked ''Deny
Loan/Reservation''.\\

}

\end{lyxlist}
\hrule

%%%%%%%%%%

\vspace{0.5cm}
\hrule
\begin{lyxlist}{PC1}
\small{
\item [\textbf{Procedure:}] Check messages.
\item [\textbf{Scope:}] LibrarianSystem (\emph{LibSys})
\item [\textbf{Primary Actor}:] Librarian Smith
\item [\textbf{Secondary Actor(s)}:] 
\item [\textbf{Goal:}] The intention of the User is to check his messages
exchanged with the system administrator.
\item [\textbf{Level}:] User-goal level
\item [\textbf{Main~Success~Scenario}]:\\
1. \emph{Smith} is in any database or the homescreen.\\
2. \emph{Smith} clicks the button ''Messages''.\\
3. \emph{Smith} now sees all the messages exchanged between him and the
system administrator.\\

}

\end{lyxlist}
\hrule


\section{Librarians}
\label{chap:usage_guide}

\subsection{Multi-procedures}

\vspace{0.5cm}
\hrule
\begin{lyxlist}{PC1}
\small{
\item [\textbf{Procedure:}] Librarian login
\item [\textbf{Scope:}] LibrarianSystem (\emph{LibSys})
\item [\textbf{Primary Actor}:] Librarian Smith
\item [\textbf{Secondary Actor(s)}:] LoginSystem LS
\item [\textbf{Goal:}] The intention of the User is to login into the
system in order to use the services of the application.
\item [\textbf{Level}:] User-goal level
\item [\textbf{Main~Success~Scenario}]:\\
1. \emph{Smith} enters the user name and password in the respective fields\\
2. \emph{Smith} presses on the login button which sends a login request message
to \emph{LS}\\
3. \emph{LS} logs \emph{Smith} into the application\\


\item [\textbf{Extensions}]:\\
3.a Username of \emph{Smith} not found\\
\hspace*{0.5cm} 3.a.1 \emph{LS} returns a message stating that the user name
or password is not correct \\
\hspace*{0.5cm} \textbf{procedure returns to step 1}

3.b Password of \emph{Smith} not correct \\
\hspace*{0.5cm} 3.b.1 \emph{LS} returns a message stating that the user name
or password is not correct \\
\hspace*{0.5cm} \textbf{procedure returns to step 1}

}

\end{lyxlist}
\hrule

\vspace{0.5cm}
\hrule
\begin{lyxlist}{PC1}
\small{
\item [\textbf{Procedure:}] Grant book loan
\item [\textbf{Scope:}] LibrarianSystem (\emph{LibSys})
\item [\textbf{Primary Actor}:] Librarian Smith
\item [\textbf{Secondary Actor(s)}:] Customer John
\item [\textbf{Goal:}] The intention of the User is to grant a loan for a book
requested by a customer.
\item [\textbf{Level}:] User-goal level
\item [\textbf{Main~Success~Scenario}]:\\
1. \emph{John} is approaching the librarian with a book he wants to loan. \\
2. \emph{John} gives the librarian his ID so the librarian can look up his
account\\
3. \emph{Smith} opens up \emph{John}'s account and clicks on ''Add loan''. \\
4. \emph{Smith} enters the book ID in the popup window and clicks on ''Add
Loan''.\\

\item [\textbf{Extensions}]:\\
2.a The librarian cannot verify the customer account\\
3.a \emph{Smith} decides not to allow the loan, either due to too many active
loans, or because a suspended or banned account.\\
\hspace*{0.5cm} 3.a.1 \emph{John} gets told by \emph{Smith} that he cannot
currently loan a book and why.\\ 
\hspace*{0.5cm} \textbf{procedure \emph{ends}}

}

\end{lyxlist}
\hrule

%

\vspace{0.5cm}
\hrule
\begin{lyxlist}{PC1}
\small{
\item [\textbf{Procedure:}] Close book loan
\item [\textbf{Scope:}] LibrarianSystem (\emph{LibSys})
\item [\textbf{Primary Actor}:] Librarian Smith
\item [\textbf{Secondary Actor(s)}:] Customer John
\item [\textbf{Goal:}] The intention of the User is to close a loan for a book
returned by a customer.
\item [\textbf{Level}:] User-goal level
\item [\textbf{Main~Success~Scenario}]:\\
1. \emph{John} is approaching the librarian with a book he wants to return. \\
2. \emph{John} gives the librarian his ID so the librarian can look up his
account\\
3. \emph{Smith} opens up \emph{John}'s account and selects the book \emph{John}
wants to return.\\
4. \emph{Smith} then clicks on ''Return Book''.\\

\item [\textbf{Extensions}]:\\
2.a The librarian cannot verify the customer account\\
3.a \emph{Smith} sees that either the book \emph{John} wants to return is
overdue and John needs to pay a fine, or due to too many overdues his account
has been suspended/banned.\\
\hspace*{0.5cm} 3.a.1 \emph{John} gets told by \emph{Smith} that he needs to
pay fine since the books is overdue or his account is suspended/banned.\\
\hspace*{0.5cm} \textbf{procedure \emph{ends}}

}

\end{lyxlist}
\hrule

%%
%%

\vspace{0.5cm}
\hrule
\begin{lyxlist}{PC1}
\small{
\item [\textbf{Procedure:}] Suspend Customer
\item [\textbf{Scope:}] LibrarianSystem (\emph{LibSys})
\item [\textbf{Primary Actor}:] Librarian Smith
\item [\textbf{Secondary Actor(s)}:] Customer John
\item [\textbf{Goal:}] The intention of the User is to suspend the customer
account until a certain fine is paid.
\item [\textbf{Level}:] User-goal level
\item [\textbf{Main~Success~Scenario}]:\\
1. \emph{Smith} has been made aware of a customer with multiple
overdue books.\\
2. \emph{Smith} opens this customers account and clicks on ''Suspend
Customer''\\
3. \emph{Smith} enters a small descriptive note in the text field on the
customer page.\\
4. \emph{John} now can't loan any more books until the fine is paid off and
every librarian accessign his account sees what the fine is.\\

}

\end{lyxlist}
\hrule

\vspace{0.5cm}
\hrule
\begin{lyxlist}{PC1}
\small{
\item [\textbf{Procedure:}] Reinstate Customer
\item [\textbf{Scope:}] LibrarianSystem (\emph{LibSys})
\item [\textbf{Primary Actor}:] Librarian Smith
\item [\textbf{Secondary Actor(s)}:] Customer John
\item [\textbf{Goal:}] The intention of the User is to reinstate a suspended
customer account.
\item [\textbf{Level}:] User-goal level
\item [\textbf{Main~Success~Scenario}]:\\
1. \emph{John} has been made aware that his customer account got suspended
and he needs to pay a fine to reinstate it.\\
2. \emph{John} goes to the library and gives \emph{Smith} his customer ID so
he can pull up his account.\\
3. \emph{Smith} opens this customers account and checks the fine \emph{John}
needs to pay.\\
4. \emph{Jonh} pays \emph{Smith} the fine.\\
5. \emph{Smith} receives the payment and clicks on ''Reinstate Account''.\\

\item [\textbf{Extensions}]:\\
4.a \emph{John} doesn't have enough money on him to pay the fine and thus
cannot complete the reinstation of his customer account.\\
}

\end{lyxlist}
\hrule

\vspace{0.5cm}
\hrule
\begin{lyxlist}{PC1}
\small{
\item [\textbf{Procedure:}] Ban Customer
\item [\textbf{Scope:}] LibrarianSystem (\emph{LibSys})
\item [\textbf{Primary Actor}:] Librarian Smith
\item [\textbf{Secondary Actor(s)}:] LibrarianSupervisor Jane, SysAdmin Fil
\item [\textbf{Goal:}] The intention of the User is to put in a request to ban a
customer.
\item [\textbf{Level}:] User-goal level
\item [\textbf{Main~Success~Scenario}]:\\
1. \emph{Smith} has been made aware of a deviant customer with multiple
infractions.\\
2. \emph{Smith} opens this customers account and click on ''Ban Customer''\\
3. \emph{Smith} enters a small descriptive text and clicks ''Send''.\\
4. \emph{Jane} receives this request and approves this.\\
5. \emph{Fil} sees the request has been approved and closes the customer
account.\\

\item [\textbf{Extensions}]:\\
4.a \emph{Jane} denies the request to close the account and thus \emph{Fil}
never gets involved.\\
}

\end{lyxlist}
\hrule

\vspace{0.5cm}
\hrule
\begin{lyxlist}{PC1}
\small{
\item [\textbf{Procedure:}] Unban Customer
\item [\textbf{Scope:}] LibrarianSystem (\emph{LibSys})
\item [\textbf{Primary Actor}:] Librarian Smith
\item [\textbf{Secondary Actor(s)}:] LibrarianSupervisor Jane, SysAdmin Fil,
Customer John
\item [\textbf{Goal:}] The intention of the User is to put in a request to unban
a customer.
\item [\textbf{Level}:] User-goal level
\item [\textbf{Main~Success~Scenario}]:\\
1. \emph{John} has paid off all his fines at the library and his ban has been
in effect for quite some time and he wants to be unbanned and promises to
follow the rules.\\
2. \emph{Smith} opens this customers account and click on ''Unban Customer''\\
3. \emph{Smith} enters a small descriptive text and clicks ''Send''.\\
4. \emph{Jane} receives this request and approves this.\\
5. \emph{Fil} sees the request has been approved and reopens the customer
account.\\

\item [\textbf{Extensions}]:\\
1.a \emph{Smith} sees that \emph{John} actually hasn't paid of all his fines or
for whatever reason doesn't think \emph{John} shouldn't be allowed to put in a
request to reopen his account.\\
4.a \emph{Jane} denies the request to reopen the account and thus \emph{Fil}
never gets involved.\\
}

\end{lyxlist}
\hrule

%%%%%%%%%%

\vspace{0.5cm}
\hrule
\begin{lyxlist}{PC1}
\small{
\item [\textbf{Procedure:}] Send message.
\item [\textbf{Scope:}] LibrarianSystem (\emph{LibSys})
\item [\textbf{Primary Actor}:] Librarian Smith
\item [\textbf{Secondary Actor(s)}:] SysAdmin Fil
\item [\textbf{Goal:}] The intention of the User is to send a message to the
system administrator.
\item [\textbf{Level}:] User-goal level
\item [\textbf{Main~Success~Scenario}]:\\
1. \emph{Smith} is in any database or the homescreen.\\
2. \emph{Smith} clicks the button ''Messages''.\\
3. \emph{Smith} moves his cursor into the text field below the messages.\\
4. \emph{Smith} enters his message for \emph{Fil} into the field.\\
5. \emph{Smith} now clicks on ''Send''.

}

\end{lyxlist}
\hrule

\subsection{Mono-procedures}

\vspace{0.5cm}
\hrule
\begin{lyxlist}{PC1}
\small{
\item [\textbf{Procedure:}] Search books list
\item [\textbf{Scope:}] LibrarianSystem (\emph{LibSys})
\item [\textbf{Primary Actor}:] Librarian Smith
\item [\textbf{Secondary Actor(s)}:] 
\item [\textbf{Goal:}] The intention of the User is to search the book database
for a specific book.
\item [\textbf{Level}:] User-goal level
\item [\textbf{Main~Success~Scenario}]:\\
1. \emph{Smith} is at the home panel of the application after logging in.\\
2. \emph{Smith} now clicks on the button ''Books'' to get to the book database
and search for books.\\
3. In the search field on top \emph{Smith} enters either the book name, ISBN or
author of the book he wants to find.\\
4. In the dropdown menu next to the search field \emph{Smith} chooses which
property he just entered.\\
5. \emph{Smith} presses ''Search'' to get a list of all the books matching his
search query.\\


\item [\textbf{Extensions}]:\\
5.a \emph{Smith} left the search field blank, which then prints him a list of
all the books in the book database.\\
}

%%

\end{lyxlist}
\hrule

\vspace{0.5cm}
\hrule
\begin{lyxlist}{PC1}
\small{
\item [\textbf{Procedure:}] Search customer list
\item [\textbf{Scope:}] LibrarianSystem (\emph{LibSys})
\item [\textbf{Primary Actor}:] Librarian Smith
\item [\textbf{Secondary Actor(s)}:] 
\item [\textbf{Goal:}] The intention of the User is to search the customer
database for a specific customer.
\item [\textbf{Level}:] User-goal level
\item [\textbf{Main~Success~Scenario}]:\\
1. \emph{Smith} is at the home panel of the application after logging in.\\
2. \emph{Smith} now clicks on the button ''Customers'' to get to the customer
database and search for customers.\\
3. In the search field on top \emph{Smith} enters either the first name, last
name, customer ID or account status of the customer he wants to find.\\
4. In the dropdown menu next to the search field \emph{Smith} chooses which
property he just entered.\\
5. \emph{Smith} presses ''Search'' to get a list of all the customers matching
his search query.\\


\item [\textbf{Extensions}]:\\
5.a \emph{Smith} left the search field blank, which then prints him a list of
all the customers in the customer database.\\
}

\end{lyxlist}
\hrule

%%

\vspace{0.5cm}
\hrule
\begin{lyxlist}{PC1}
\small{
\item [\textbf{Procedure:}] Inspect customer
\item [\textbf{Scope:}] LibrarianSystem (\emph{LibSys})
\item [\textbf{Primary Actor}:] Librarian Smith
\item [\textbf{Secondary Actor(s)}:] 
\item [\textbf{Goal:}] The intention of the User is to inspect a customers
profile to get further information about current or past loans, comments from
other librarians, or take further actions with his account.
\item [\textbf{Level}:] User-goal level
\item [\textbf{Main~Success~Scenario}]:\\
1. \emph{Smith} successfully launched a search query on the customer database.\\
2. \emph{Smith} now selects the customer he wishes to inspect from the list
displayed in the application matching his search query.\\
3. A new window will popup displaying per default current loans, with the
possibility to show past loans instead, as well as giving the librarian
the possibility to grant a loan, make a book return, suspend or reinstate the
account, request a ban or unban and add notes to his page.\\


\item [\textbf{Extensions}]:\\
}

\end{lyxlist}
\hrule

%%
%%%%%%%%%%%%%%%%%%%%%%%%

\vspace{0.5cm}
\hrule
\begin{lyxlist}{PC1}
\small{
\item [\textbf{Procedure:}] Show loan history
\item [\textbf{Scope:}] LibrarianSystem (\emph{LibSys})
\item [\textbf{Primary Actor}:] Librarian Smith
\item [\textbf{Secondary Actor(s)}:] 
\item [\textbf{Goal:}] The intention of the User is to check a customers loan
history
\item [\textbf{Level}:] User-goal level
\item [\textbf{Main~Success~Scenario}]:\\
1. \emph{Smith} is at the account page of a customer.\\
2. \emph{Smith} clicks the button on the bottom right titled ''History'' which
replaces the current table of current loans which a table of past loans as well
as replacing the button with another one to get back to current loans.


\item [\textbf{Extensions}]:\\
}

\end{lyxlist}
\hrule


%%%%%%%%%%%%%%%%%%%%%%

\vspace{0.5cm}
\hrule
\begin{lyxlist}{PC1}
\small{
\item [\textbf{Procedure:}] Edit customer notes.
\item [\textbf{Scope:}] LibrarianSystem (\emph{LibSys})
\item [\textbf{Primary Actor}:] Librarian Smith
\item [\textbf{Secondary Actor(s)}:] 
\item [\textbf{Goal:}] The intention of the User is to edit the notes and
comments on a customers account page.
\item [\textbf{Level}:] User-goal level
\item [\textbf{Main~Success~Scenario}]:\\
1. \emph{Smith} is at the account page of a customer.\\
2. \emph{Smith} clicks in the text box on the bottom of the window and adds his
notes and comments on the customers in this field like fines to be paid or
suspension issued at date DD/MM/YYYY.


\item [\textbf{Extensions}]:\\
}

\end{lyxlist}
\hrule

%%%%%%%%%%%%%%%%%%%

\vspace{0.5cm}
\hrule
\begin{lyxlist}{PC1}
\small{
\item [\textbf{Procedure:}] Add book.
\item [\textbf{Scope:}] LibrarianSystem (\emph{LibSys})
\item [\textbf{Primary Actor}:] Librarian Smith
\item [\textbf{Secondary Actor(s)}:] 
\item [\textbf{Goal:}] The intention of the User is to add a book to the
book database.
\item [\textbf{Level}:] User-goal level
\item [\textbf{Main~Success~Scenario}]:\\
1. \emph{Smith} is at the book database menu.\\
2. \emph{Smith} clicks on the button labeled ''Add Book''.\\
3. \emph{Smith} enters the books ISBN into the popup window and clicks on ''Add
Book''.\\
4. The system now automatically adds a new entry for this book, looking up the
author and title associated with the ISBN and generating a new book ID for this
book.\\


\item [\textbf{Extensions}]:\\
3.a \emph{Smith} entered a wrong ISBN prompting him to either cancel the
procedure or manually adding book name and author name.\\
}

\end{lyxlist}
\hrule


%%%%%%%%%%

\vspace{0.5cm}
\hrule
\begin{lyxlist}{PC1}
\small{
\item [\textbf{Procedure:}] Remove book.
\item [\textbf{Scope:}] LibrarianSystem (\emph{LibSys})
\item [\textbf{Primary Actor}:] Librarian Smith
\item [\textbf{Secondary Actor(s)}:] 
\item [\textbf{Goal:}] The intention of the User is to remove a book from the
book database.
\item [\textbf{Level}:] User-goal level
\item [\textbf{Main~Success~Scenario}]:\\
1. \emph{Smith} is at the book database menu.\\
2. \emph{Smith} selects the book he wishes to remove from the book database.\\
3. \emph{Smith} clicks on the button labeled ''Remove Book''.\\
4. The system now removes this particular copy of a book from the
database.\\


\item [\textbf{Extensions}]:\\
3.a \emph{Smith} chose a book currently being loaned which will bring up a
prompt telling him he can't remove this copy from the database.\\
3.b \emph{Smith} chose a book currently being reserved by a customer which will
bring up a prompt if he really wants to remove this book from the database as
this will cancel the reservation.\\
}

\end{lyxlist}
\hrule

%%%%%%%%%%

\vspace{0.5cm}
\hrule
\begin{lyxlist}{PC1}
\small{
\item [\textbf{Procedure:}] Check Loan\&Reservation requests.
\item [\textbf{Scope:}] LibrarianSystem (\emph{LibSys})
\item [\textbf{Primary Actor}:] Librarian Smith
\item [\textbf{Secondary Actor(s)}:] 
\item [\textbf{Goal:}] The intention of the User is to check loan and
reservation requests made via the mobile app.
\item [\textbf{Level}:] User-goal level
\item [\textbf{Main~Success~Scenario}]:\\
1. \emph{Smith} is in any database or the homescreen.\\
2. \emph{Smith} clicks the button ''Loan\&Reservation requests''.\\
3. \emph{Smith} now sees all loan and reservation requests made via the
mobile app and decide whether or not to grant them.\\

}

\end{lyxlist}
\hrule

%%%%%%%%%%

\vspace{0.5cm}
\hrule
\begin{lyxlist}{PC1}
\small{
\item [\textbf{Procedure:}] Grant Loan\&Reservation requests.
\item [\textbf{Scope:}] LibrarianSystem (\emph{LibSys})
\item [\textbf{Primary Actor}:] Librarian Smith
\item [\textbf{Secondary Actor(s)}:] 
\item [\textbf{Goal:}] The intention of the User is to grant a loan or
reservation requests made via the mobile app.
\item [\textbf{Level}:] User-goal level
\item [\textbf{Main~Success~Scenario}]:\\
1. \emph{Smith} is in any database or the homescreen.\\
2. \emph{Smith} clicks the button ''Loan\&Reservation requests''.\\
3. \emph{Smith} selects one of the requests.\\
4. \emph{Smith} now presses the button below marked ''Grant
Loan/Reservation''.\\

}

\end{lyxlist}
\hrule

%%%%%%%%%%

\vspace{0.5cm}
\hrule
\begin{lyxlist}{PC1}
\small{
\item [\textbf{Procedure:}] Deny Loan\&Reservation requests.
\item [\textbf{Scope:}] LibrarianSystem (\emph{LibSys})
\item [\textbf{Primary Actor}:] Librarian Smith
\item [\textbf{Secondary Actor(s)}:] 
\item [\textbf{Goal:}] The intention of the User is to deny a loan or
reservation requests made via the mobile app.
\item [\textbf{Level}:] User-goal level
\item [\textbf{Main~Success~Scenario}]:\\
1. \emph{Smith} is in any database or the homescreen.\\
2. \emph{Smith} clicks the button ''Loan\&Reservation requests''.\\
3. \emph{Smith} selects one of the requests.\\
4. \emph{Smith} now presses the button below marked ''Deny
Loan/Reservation''.\\

}

\end{lyxlist}
\hrule

%%%%%%%%%%

\vspace{0.5cm}
\hrule
\begin{lyxlist}{PC1}
\small{
\item [\textbf{Procedure:}] Check messages.
\item [\textbf{Scope:}] LibrarianSystem (\emph{LibSys})
\item [\textbf{Primary Actor}:] Librarian Smith
\item [\textbf{Secondary Actor(s)}:] 
\item [\textbf{Goal:}] The intention of the User is to check his messages
exchanged with the system administrator.
\item [\textbf{Level}:] User-goal level
\item [\textbf{Main~Success~Scenario}]:\\
1. \emph{Smith} is in any database or the homescreen.\\
2. \emph{Smith} clicks the button ''Messages''.\\
3. \emph{Smith} now sees all the messages exchanged between him and the
system administrator.\\

}

\end{lyxlist}
\hrule


\section{Librarians}
\label{chap:usage_guide}

\subsection{Multi-procedures}

\vspace{0.5cm}
\hrule
\begin{lyxlist}{PC1}
\small{
\item [\textbf{Procedure:}] Librarian login
\item [\textbf{Scope:}] LibrarianSystem (\emph{LibSys})
\item [\textbf{Primary Actor}:] Librarian Smith
\item [\textbf{Secondary Actor(s)}:] LoginSystem LS
\item [\textbf{Goal:}] The intention of the User is to login into the
system in order to use the services of the application.
\item [\textbf{Level}:] User-goal level
\item [\textbf{Main~Success~Scenario}]:\\
1. \emph{Smith} enters the user name and password in the respective fields\\
2. \emph{Smith} presses on the login button which sends a login request message
to \emph{LS}\\
3. \emph{LS} logs \emph{Smith} into the application\\


\item [\textbf{Extensions}]:\\
3.a Username of \emph{Smith} not found\\
\hspace*{0.5cm} 3.a.1 \emph{LS} returns a message stating that the user name
or password is not correct \\
\hspace*{0.5cm} \textbf{procedure returns to step 1}

3.b Password of \emph{Smith} not correct \\
\hspace*{0.5cm} 3.b.1 \emph{LS} returns a message stating that the user name
or password is not correct \\
\hspace*{0.5cm} \textbf{procedure returns to step 1}

}

\end{lyxlist}
\hrule

\vspace{0.5cm}
\hrule
\begin{lyxlist}{PC1}
\small{
\item [\textbf{Procedure:}] Grant book loan
\item [\textbf{Scope:}] LibrarianSystem (\emph{LibSys})
\item [\textbf{Primary Actor}:] Librarian Smith
\item [\textbf{Secondary Actor(s)}:] Customer John
\item [\textbf{Goal:}] The intention of the User is to grant a loan for a book
requested by a customer.
\item [\textbf{Level}:] User-goal level
\item [\textbf{Main~Success~Scenario}]:\\
1. \emph{John} is approaching the librarian with a book he wants to loan. \\
2. \emph{John} gives the librarian his ID so the librarian can look up his
account\\
3. \emph{Smith} opens up \emph{John}'s account and clicks on ''Add loan''. \\
4. \emph{Smith} enters the book ID in the popup window and clicks on ''Add
Loan''.\\

\item [\textbf{Extensions}]:\\
2.a The librarian cannot verify the customer account\\
3.a \emph{Smith} decides not to allow the loan, either due to too many active
loans, or because a suspended or banned account.\\
\hspace*{0.5cm} 3.a.1 \emph{John} gets told by \emph{Smith} that he cannot
currently loan a book and why.\\ 
\hspace*{0.5cm} \textbf{procedure \emph{ends}}

}

\end{lyxlist}
\hrule

%

\vspace{0.5cm}
\hrule
\begin{lyxlist}{PC1}
\small{
\item [\textbf{Procedure:}] Close book loan
\item [\textbf{Scope:}] LibrarianSystem (\emph{LibSys})
\item [\textbf{Primary Actor}:] Librarian Smith
\item [\textbf{Secondary Actor(s)}:] Customer John
\item [\textbf{Goal:}] The intention of the User is to close a loan for a book
returned by a customer.
\item [\textbf{Level}:] User-goal level
\item [\textbf{Main~Success~Scenario}]:\\
1. \emph{John} is approaching the librarian with a book he wants to return. \\
2. \emph{John} gives the librarian his ID so the librarian can look up his
account\\
3. \emph{Smith} opens up \emph{John}'s account and selects the book \emph{John}
wants to return.\\
4. \emph{Smith} then clicks on ''Return Book''.\\

\item [\textbf{Extensions}]:\\
2.a The librarian cannot verify the customer account\\
3.a \emph{Smith} sees that either the book \emph{John} wants to return is
overdue and John needs to pay a fine, or due to too many overdues his account
has been suspended/banned.\\
\hspace*{0.5cm} 3.a.1 \emph{John} gets told by \emph{Smith} that he needs to
pay fine since the books is overdue or his account is suspended/banned.\\
\hspace*{0.5cm} \textbf{procedure \emph{ends}}

}

\end{lyxlist}
\hrule

%%
%%

\vspace{0.5cm}
\hrule
\begin{lyxlist}{PC1}
\small{
\item [\textbf{Procedure:}] Suspend Customer
\item [\textbf{Scope:}] LibrarianSystem (\emph{LibSys})
\item [\textbf{Primary Actor}:] Librarian Smith
\item [\textbf{Secondary Actor(s)}:] Customer John
\item [\textbf{Goal:}] The intention of the User is to suspend the customer
account until a certain fine is paid.
\item [\textbf{Level}:] User-goal level
\item [\textbf{Main~Success~Scenario}]:\\
1. \emph{Smith} has been made aware of a customer with multiple
overdue books.\\
2. \emph{Smith} opens this customers account and clicks on ''Suspend
Customer''\\
3. \emph{Smith} enters a small descriptive note in the text field on the
customer page.\\
4. \emph{John} now can't loan any more books until the fine is paid off and
every librarian accessign his account sees what the fine is.\\

}

\end{lyxlist}
\hrule

\vspace{0.5cm}
\hrule
\begin{lyxlist}{PC1}
\small{
\item [\textbf{Procedure:}] Reinstate Customer
\item [\textbf{Scope:}] LibrarianSystem (\emph{LibSys})
\item [\textbf{Primary Actor}:] Librarian Smith
\item [\textbf{Secondary Actor(s)}:] Customer John
\item [\textbf{Goal:}] The intention of the User is to reinstate a suspended
customer account.
\item [\textbf{Level}:] User-goal level
\item [\textbf{Main~Success~Scenario}]:\\
1. \emph{John} has been made aware that his customer account got suspended
and he needs to pay a fine to reinstate it.\\
2. \emph{John} goes to the library and gives \emph{Smith} his customer ID so
he can pull up his account.\\
3. \emph{Smith} opens this customers account and checks the fine \emph{John}
needs to pay.\\
4. \emph{Jonh} pays \emph{Smith} the fine.\\
5. \emph{Smith} receives the payment and clicks on ''Reinstate Account''.\\

\item [\textbf{Extensions}]:\\
4.a \emph{John} doesn't have enough money on him to pay the fine and thus
cannot complete the reinstation of his customer account.\\
}

\end{lyxlist}
\hrule

\vspace{0.5cm}
\hrule
\begin{lyxlist}{PC1}
\small{
\item [\textbf{Procedure:}] Ban Customer
\item [\textbf{Scope:}] LibrarianSystem (\emph{LibSys})
\item [\textbf{Primary Actor}:] Librarian Smith
\item [\textbf{Secondary Actor(s)}:] LibrarianSupervisor Jane, SysAdmin Fil
\item [\textbf{Goal:}] The intention of the User is to put in a request to ban a
customer.
\item [\textbf{Level}:] User-goal level
\item [\textbf{Main~Success~Scenario}]:\\
1. \emph{Smith} has been made aware of a deviant customer with multiple
infractions.\\
2. \emph{Smith} opens this customers account and click on ''Ban Customer''\\
3. \emph{Smith} enters a small descriptive text and clicks ''Send''.\\
4. \emph{Jane} receives this request and approves this.\\
5. \emph{Fil} sees the request has been approved and closes the customer
account.\\

\item [\textbf{Extensions}]:\\
4.a \emph{Jane} denies the request to close the account and thus \emph{Fil}
never gets involved.\\
}

\end{lyxlist}
\hrule

\vspace{0.5cm}
\hrule
\begin{lyxlist}{PC1}
\small{
\item [\textbf{Procedure:}] Unban Customer
\item [\textbf{Scope:}] LibrarianSystem (\emph{LibSys})
\item [\textbf{Primary Actor}:] Librarian Smith
\item [\textbf{Secondary Actor(s)}:] LibrarianSupervisor Jane, SysAdmin Fil,
Customer John
\item [\textbf{Goal:}] The intention of the User is to put in a request to unban
a customer.
\item [\textbf{Level}:] User-goal level
\item [\textbf{Main~Success~Scenario}]:\\
1. \emph{John} has paid off all his fines at the library and his ban has been
in effect for quite some time and he wants to be unbanned and promises to
follow the rules.\\
2. \emph{Smith} opens this customers account and click on ''Unban Customer''\\
3. \emph{Smith} enters a small descriptive text and clicks ''Send''.\\
4. \emph{Jane} receives this request and approves this.\\
5. \emph{Fil} sees the request has been approved and reopens the customer
account.\\

\item [\textbf{Extensions}]:\\
1.a \emph{Smith} sees that \emph{John} actually hasn't paid of all his fines or
for whatever reason doesn't think \emph{John} shouldn't be allowed to put in a
request to reopen his account.\\
4.a \emph{Jane} denies the request to reopen the account and thus \emph{Fil}
never gets involved.\\
}

\end{lyxlist}
\hrule

%%%%%%%%%%

\vspace{0.5cm}
\hrule
\begin{lyxlist}{PC1}
\small{
\item [\textbf{Procedure:}] Send message.
\item [\textbf{Scope:}] LibrarianSystem (\emph{LibSys})
\item [\textbf{Primary Actor}:] Librarian Smith
\item [\textbf{Secondary Actor(s)}:] SysAdmin Fil
\item [\textbf{Goal:}] The intention of the User is to send a message to the
system administrator.
\item [\textbf{Level}:] User-goal level
\item [\textbf{Main~Success~Scenario}]:\\
1. \emph{Smith} is in any database or the homescreen.\\
2. \emph{Smith} clicks the button ''Messages''.\\
3. \emph{Smith} moves his cursor into the text field below the messages.\\
4. \emph{Smith} enters his message for \emph{Fil} into the field.\\
5. \emph{Smith} now clicks on ''Send''.

}

\end{lyxlist}
\hrule

\subsection{Mono-procedures}

\vspace{0.5cm}
\hrule
\begin{lyxlist}{PC1}
\small{
\item [\textbf{Procedure:}] Search books list
\item [\textbf{Scope:}] LibrarianSystem (\emph{LibSys})
\item [\textbf{Primary Actor}:] Librarian Smith
\item [\textbf{Secondary Actor(s)}:] 
\item [\textbf{Goal:}] The intention of the User is to search the book database
for a specific book.
\item [\textbf{Level}:] User-goal level
\item [\textbf{Main~Success~Scenario}]:\\
1. \emph{Smith} is at the home panel of the application after logging in.\\
2. \emph{Smith} now clicks on the button ''Books'' to get to the book database
and search for books.\\
3. In the search field on top \emph{Smith} enters either the book name, ISBN or
author of the book he wants to find.\\
4. In the dropdown menu next to the search field \emph{Smith} chooses which
property he just entered.\\
5. \emph{Smith} presses ''Search'' to get a list of all the books matching his
search query.\\


\item [\textbf{Extensions}]:\\
5.a \emph{Smith} left the search field blank, which then prints him a list of
all the books in the book database.\\
}

%%

\end{lyxlist}
\hrule

\vspace{0.5cm}
\hrule
\begin{lyxlist}{PC1}
\small{
\item [\textbf{Procedure:}] Search customer list
\item [\textbf{Scope:}] LibrarianSystem (\emph{LibSys})
\item [\textbf{Primary Actor}:] Librarian Smith
\item [\textbf{Secondary Actor(s)}:] 
\item [\textbf{Goal:}] The intention of the User is to search the customer
database for a specific customer.
\item [\textbf{Level}:] User-goal level
\item [\textbf{Main~Success~Scenario}]:\\
1. \emph{Smith} is at the home panel of the application after logging in.\\
2. \emph{Smith} now clicks on the button ''Customers'' to get to the customer
database and search for customers.\\
3. In the search field on top \emph{Smith} enters either the first name, last
name, customer ID or account status of the customer he wants to find.\\
4. In the dropdown menu next to the search field \emph{Smith} chooses which
property he just entered.\\
5. \emph{Smith} presses ''Search'' to get a list of all the customers matching
his search query.\\


\item [\textbf{Extensions}]:\\
5.a \emph{Smith} left the search field blank, which then prints him a list of
all the customers in the customer database.\\
}

\end{lyxlist}
\hrule

%%

\vspace{0.5cm}
\hrule
\begin{lyxlist}{PC1}
\small{
\item [\textbf{Procedure:}] Inspect customer
\item [\textbf{Scope:}] LibrarianSystem (\emph{LibSys})
\item [\textbf{Primary Actor}:] Librarian Smith
\item [\textbf{Secondary Actor(s)}:] 
\item [\textbf{Goal:}] The intention of the User is to inspect a customers
profile to get further information about current or past loans, comments from
other librarians, or take further actions with his account.
\item [\textbf{Level}:] User-goal level
\item [\textbf{Main~Success~Scenario}]:\\
1. \emph{Smith} successfully launched a search query on the customer database.\\
2. \emph{Smith} now selects the customer he wishes to inspect from the list
displayed in the application matching his search query.\\
3. A new window will popup displaying per default current loans, with the
possibility to show past loans instead, as well as giving the librarian
the possibility to grant a loan, make a book return, suspend or reinstate the
account, request a ban or unban and add notes to his page.\\


\item [\textbf{Extensions}]:\\
}

\end{lyxlist}
\hrule

%%
%%%%%%%%%%%%%%%%%%%%%%%%

\vspace{0.5cm}
\hrule
\begin{lyxlist}{PC1}
\small{
\item [\textbf{Procedure:}] Show loan history
\item [\textbf{Scope:}] LibrarianSystem (\emph{LibSys})
\item [\textbf{Primary Actor}:] Librarian Smith
\item [\textbf{Secondary Actor(s)}:] 
\item [\textbf{Goal:}] The intention of the User is to check a customers loan
history
\item [\textbf{Level}:] User-goal level
\item [\textbf{Main~Success~Scenario}]:\\
1. \emph{Smith} is at the account page of a customer.\\
2. \emph{Smith} clicks the button on the bottom right titled ''History'' which
replaces the current table of current loans which a table of past loans as well
as replacing the button with another one to get back to current loans.


\item [\textbf{Extensions}]:\\
}

\end{lyxlist}
\hrule


%%%%%%%%%%%%%%%%%%%%%%

\vspace{0.5cm}
\hrule
\begin{lyxlist}{PC1}
\small{
\item [\textbf{Procedure:}] Edit customer notes.
\item [\textbf{Scope:}] LibrarianSystem (\emph{LibSys})
\item [\textbf{Primary Actor}:] Librarian Smith
\item [\textbf{Secondary Actor(s)}:] 
\item [\textbf{Goal:}] The intention of the User is to edit the notes and
comments on a customers account page.
\item [\textbf{Level}:] User-goal level
\item [\textbf{Main~Success~Scenario}]:\\
1. \emph{Smith} is at the account page of a customer.\\
2. \emph{Smith} clicks in the text box on the bottom of the window and adds his
notes and comments on the customers in this field like fines to be paid or
suspension issued at date DD/MM/YYYY.


\item [\textbf{Extensions}]:\\
}

\end{lyxlist}
\hrule

%%%%%%%%%%%%%%%%%%%

\vspace{0.5cm}
\hrule
\begin{lyxlist}{PC1}
\small{
\item [\textbf{Procedure:}] Add book.
\item [\textbf{Scope:}] LibrarianSystem (\emph{LibSys})
\item [\textbf{Primary Actor}:] Librarian Smith
\item [\textbf{Secondary Actor(s)}:] 
\item [\textbf{Goal:}] The intention of the User is to add a book to the
book database.
\item [\textbf{Level}:] User-goal level
\item [\textbf{Main~Success~Scenario}]:\\
1. \emph{Smith} is at the book database menu.\\
2. \emph{Smith} clicks on the button labeled ''Add Book''.\\
3. \emph{Smith} enters the books ISBN into the popup window and clicks on ''Add
Book''.\\
4. The system now automatically adds a new entry for this book, looking up the
author and title associated with the ISBN and generating a new book ID for this
book.\\


\item [\textbf{Extensions}]:\\
3.a \emph{Smith} entered a wrong ISBN prompting him to either cancel the
procedure or manually adding book name and author name.\\
}

\end{lyxlist}
\hrule


%%%%%%%%%%

\vspace{0.5cm}
\hrule
\begin{lyxlist}{PC1}
\small{
\item [\textbf{Procedure:}] Remove book.
\item [\textbf{Scope:}] LibrarianSystem (\emph{LibSys})
\item [\textbf{Primary Actor}:] Librarian Smith
\item [\textbf{Secondary Actor(s)}:] 
\item [\textbf{Goal:}] The intention of the User is to remove a book from the
book database.
\item [\textbf{Level}:] User-goal level
\item [\textbf{Main~Success~Scenario}]:\\
1. \emph{Smith} is at the book database menu.\\
2. \emph{Smith} selects the book he wishes to remove from the book database.\\
3. \emph{Smith} clicks on the button labeled ''Remove Book''.\\
4. The system now removes this particular copy of a book from the
database.\\


\item [\textbf{Extensions}]:\\
3.a \emph{Smith} chose a book currently being loaned which will bring up a
prompt telling him he can't remove this copy from the database.\\
3.b \emph{Smith} chose a book currently being reserved by a customer which will
bring up a prompt if he really wants to remove this book from the database as
this will cancel the reservation.\\
}

\end{lyxlist}
\hrule

%%%%%%%%%%

\vspace{0.5cm}
\hrule
\begin{lyxlist}{PC1}
\small{
\item [\textbf{Procedure:}] Check Loan\&Reservation requests.
\item [\textbf{Scope:}] LibrarianSystem (\emph{LibSys})
\item [\textbf{Primary Actor}:] Librarian Smith
\item [\textbf{Secondary Actor(s)}:] 
\item [\textbf{Goal:}] The intention of the User is to check loan and
reservation requests made via the mobile app.
\item [\textbf{Level}:] User-goal level
\item [\textbf{Main~Success~Scenario}]:\\
1. \emph{Smith} is in any database or the homescreen.\\
2. \emph{Smith} clicks the button ''Loan\&Reservation requests''.\\
3. \emph{Smith} now sees all loan and reservation requests made via the
mobile app and decide whether or not to grant them.\\

}

\end{lyxlist}
\hrule

%%%%%%%%%%

\vspace{0.5cm}
\hrule
\begin{lyxlist}{PC1}
\small{
\item [\textbf{Procedure:}] Grant Loan\&Reservation requests.
\item [\textbf{Scope:}] LibrarianSystem (\emph{LibSys})
\item [\textbf{Primary Actor}:] Librarian Smith
\item [\textbf{Secondary Actor(s)}:] 
\item [\textbf{Goal:}] The intention of the User is to grant a loan or
reservation requests made via the mobile app.
\item [\textbf{Level}:] User-goal level
\item [\textbf{Main~Success~Scenario}]:\\
1. \emph{Smith} is in any database or the homescreen.\\
2. \emph{Smith} clicks the button ''Loan\&Reservation requests''.\\
3. \emph{Smith} selects one of the requests.\\
4. \emph{Smith} now presses the button below marked ''Grant
Loan/Reservation''.\\

}

\end{lyxlist}
\hrule

%%%%%%%%%%

\vspace{0.5cm}
\hrule
\begin{lyxlist}{PC1}
\small{
\item [\textbf{Procedure:}] Deny Loan\&Reservation requests.
\item [\textbf{Scope:}] LibrarianSystem (\emph{LibSys})
\item [\textbf{Primary Actor}:] Librarian Smith
\item [\textbf{Secondary Actor(s)}:] 
\item [\textbf{Goal:}] The intention of the User is to deny a loan or
reservation requests made via the mobile app.
\item [\textbf{Level}:] User-goal level
\item [\textbf{Main~Success~Scenario}]:\\
1. \emph{Smith} is in any database or the homescreen.\\
2. \emph{Smith} clicks the button ''Loan\&Reservation requests''.\\
3. \emph{Smith} selects one of the requests.\\
4. \emph{Smith} now presses the button below marked ''Deny
Loan/Reservation''.\\

}

\end{lyxlist}
\hrule

%%%%%%%%%%

\vspace{0.5cm}
\hrule
\begin{lyxlist}{PC1}
\small{
\item [\textbf{Procedure:}] Check messages.
\item [\textbf{Scope:}] LibrarianSystem (\emph{LibSys})
\item [\textbf{Primary Actor}:] Librarian Smith
\item [\textbf{Secondary Actor(s)}:] 
\item [\textbf{Goal:}] The intention of the User is to check his messages
exchanged with the system administrator.
\item [\textbf{Level}:] User-goal level
\item [\textbf{Main~Success~Scenario}]:\\
1. \emph{Smith} is in any database or the homescreen.\\
2. \emph{Smith} clicks the button ''Messages''.\\
3. \emph{Smith} now sees all the messages exchanged between him and the
system administrator.\\

}

\end{lyxlist}
\hrule
















