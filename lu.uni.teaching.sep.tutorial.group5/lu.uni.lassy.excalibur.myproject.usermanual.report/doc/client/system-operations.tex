\section{Clients}
\label{operation:Clients}

This subsection provides a detailed description of the client
mobile application's specific functionalities.

%%%%%%%%%%%%%%%%
\subsection{Request book loan}

When browsing books, the user has the option to request a book loan. The
request is sent to the main server, where it will wait for a librarian's input, whether
the loan is allowed or not.

\begin{description}

\item \textbf{Parameters: } ISBN number, Book name

\item \textbf{Precondition: } The library has at least one physical copy
available of the required book in store

\item \textbf{Post-condition: } 

\item \textbf{Output messages: } The client receives a confirmation of his
request being sent successfully

\item \textbf{Triggering:}
\begin{enumerate}
\item  User opens the list of available books via the interface button on the
bottom of the screen.

\item  He then proceeds to click on a book of his choice.

\item In the newly opened screen, the user clicks on the Request loan button.

\item When the user clicks Request loan, the request is sent to the server and
it then awaits for further confirmation.
\end{enumerate}
\end{description}

%%%%%%%%%%%%%%%%%
\subsection{Request book reservation}

 When browsing books, the user has the option to request a reservation for a
 book that is currently unavailable.
 The request is sent to the main server, where it will wait for a copy of the
 book to be returned back, and then the loan can be confirmed.

\begin{description}

\item \textbf{Parameters: } ISBN number, Book name, Date of reservation

\item \textbf{Precondition: } The library has no physical copies available for the
said book.

\item \textbf{Post-condition:}  

\item \textbf{Output messages:} The client receives a confirmation of his
reservation request being sent successfully.

\item \textbf{Triggering:}
\begin{enumerate}

\item User opens the list of available books via the interface button on the
bottom of the screen.

\item He then proceeds to click on a book of his choice.

\item The user can clicks on the Request reservation button on the newly opened
screen.

\item When the user clicks Request loan, the request is put in a queue,
awaiting for his turn to receive a copy of the book.
\end{enumerate}
\end{description}

%%%%%%%%%%%%%%%%
\subsection{Cancel book reservation}

 Apart from being able to send a book reservation request, the user also has the
 option to cancel one. The cancelation is effective immediately.

\begin{description}

\item \textbf{Parameters:} Book name

\item \textbf{Precondition:} The user has sent at least one request for a book
reservation.

\item \textbf{Post-condition:} The reservation request is removed from the
queue in the server.

\item \textbf{Output messages:} The user receives a message saying that his
cancelation has been successful and is returned to the reservation-browsing
screen.

\item \textbf{Triggering:}
\begin{enumerate}
\item Click on the Browse loans button at the bottom of the main screen.

\item Click on the Reservation button in the top half of the screen.

\item  In the list of reservations, click on the book's reservation that is to
be cancelled.

\item In the newly opened screen, click on the Cancel reservation button.

\item The user is returned back to the reservation list screen.

\end{enumerate}

\end{description}
%%%%%%%%%%%%%%%%%
%% Payment option system operations
%%%%%%%%%%%%%%%%%

%% Set default payment option
\subsection{Set default payment option- User prompt}

 When managing payment options, it is of utmost importance, that at any given
 moment, there is at least ONE payment method registered to the user's account.
 This operation is used to automatically select a new default payment option.

\begin{description}

\item \textbf{Parameters:} User ID

\item \textbf{Precondition:} The user has exactly one payment option available
to his account.

\item \textbf{Post-condition:} The payment option is marked as the
default for the user's account.

\item \textbf{Output messages:} The user receives a message saying that his
default payment option has been updated.

\item \textbf{Triggering:}
\begin{enumerate}
\item After adding a new, or removing an old payment method that was selected as
default, the operation is initiated.

\item A request is sent to the system, asking to update the user's default
payment method.

\item The system selects the only remaining payment option and selects it as
default.

\item If update is successful, a message is sent back to the user.

\end{enumerate}

\end{description}

%% set default payment option
\subsection{Set default payment option- User prompt}

 When managing payment options, it is of utmost importance, that at any given
 moment, there is at least ONE payment method registered to the user's account.
 This operation is used to select a new default payment method after each
 modification.

\begin{description}

\item \textbf{Parameters:} User ID

\item \textbf{Precondition:} The user has at least two registered payment
options.

\item \textbf{Post-condition:} The selected payment option is marked as the
default for the user's account.

\item \textbf{Output messages:} The user receives a message saying that his
default payment option has been updated.

\item \textbf{Triggering:}
\begin{enumerate}
\item After adding a new, or removing an old payment method that was selected as
default, system prompts the user to select a new default method.

\item System lists all registered payment methods.

\item User selects one and presses the 'Submit choice' button.

\item A request is sent to the system, asking to update the user's default
payment method.

\item If update is successful, a message is sent back to the user.

\end{enumerate}

\end{description}

%% Visa adding payment option
\subsection{Add Visa payment}

 When registering and also editing his personal data, the client needs to be
 able to add a Visa card as a payment option. This operation is used in both
 instances, to handle the user's data and store it safely.

\begin{description}

\item \textbf{Parameters:} User ID, Card number, Card holder, Security number

\item \textbf{Precondition:} The user has no currently registered Visa card
connected to his account.

\item \textbf{Post-condition:} The card information is successfully added to the
database.

\item \textbf{Output messages:} The user receives a message saying that his
payment option has been registered into the system, and is asked if he wants to
make it the default payment option.

\item \textbf{Triggering:}
\begin{enumerate}
\item After entering data, press 'Add card' button.

\item The information is sent to the system server.

\item The system server sends the information to a Visa server, and confirms
data.

\item If confirmation successful, the system server stores the data in the
database.

\item A message is sent to the client, confirming the addition of his payment
option.

\end{enumerate}

\end{description}

%% Visa removing payment option
\subsection{Remove Visa payment option}

 When editing his personal data, the client needs to be
 able to remove a Visa card from his list of payment options. This operation is
 used to remove a card from the user's account in the system's database.

\begin{description}

\item \textbf{Parameters:} User ID, Card number, Card holder, Security number

\item \textbf{Precondition:} The user has already registered a valid
Visa card to his account.
\item \textbf{Precondition1 :} The user has at least one valid payment option
registered to his account, apart from the Visa option.

\item \textbf{Post-condition:} The card information is successfully removed from
the database.

\item \textbf{Output messages:} The user receives a message saying that his
payment option has been removed from the system.

\item \textbf{Triggering:}
\begin{enumerate}
\item After selecting Visa payment option, click on 'Remove'

\item A request is sent to the system server, asking to remove the Visa card
from the client's account.

\item The system removes the Visa card from the client's account and selects a
new default payment option.

\item A message is sent to the client, confirming the removal of his Visa
card from his account.

\end{enumerate}

\end{description}

%% Mastercard payment option
\subsection{Add Mastercard payment}

 When registering and also editing his personal data, the client needs to be
 able to add a Mastercard card as a payment option. This operation is used in
 both instances, to handle the user's data and store it safely.

\begin{description}

\item \textbf{Parameters:} User ID, Card number, Card holder, Security number

\item \textbf{Precondition:} The user has no currently registered Mastercard
card connected to his account.

\item \textbf{Post-condition:} The card information is successfully added to the
database.

\item \textbf{Output messages:} The user receives a message saying that his
payment option has been registered into the system, and is asked if he wants to
make it the default payment option.

\item \textbf{Triggering:}
\begin{enumerate}
\item After entering data, press 'Add card' button.

\item The information is sent to the system server.

\item The system server sends the information to a Mastercard server, and
confirms data.

\item If confirmation successful, the system server stores the data in the
database.

\item A message is sent to the client, confirming the addition of his payment
option.

\end{enumerate}

\end{description}
%% Mastercard removing option
\subsection{Remove MasterCard payment option}

 When editing his personal data, the client needs to be
 able to remove a MasterCard card from his list of payment options. This operation is
 used to remove a card from the user's account in the system's database.

\begin{description}

\item \textbf{Parameters:} User ID, Card number, Card holder, Security number

\item \textbf{Precondition:} The user has already registered a valid
MasterCard card to his account.
\item \textbf{Precondition1 :} The user has at least one valid payment option
registered to his account, apart from the MasterCard option.

\item \textbf{Post-condition:} The card information is successfully removed from
the database.

\item \textbf{Output messages:} The user receives a message saying that his
payment option has been removed from the system.

\item \textbf{Triggering:}
\begin{enumerate}
\item After selecting MasterCard payment option, click on 'Remove'

\item A request is sent to the system server, asking to remove the MasterCard card
from the client's account.

\item The system removes the MasterCard card from the client's account and selects a
new default payment option.

\item A message is sent to the client, confirming the removal of his MasterCard
card from his account.

\end{enumerate}

\end{description}

%% Paypal payment option
\subsection{Add Paypal payment}

 When registering and also editing his personal data, the client needs to be
 able to add a Paypal account as a payment option. This operation is used
 in both instances, to handle the user's data and store it safely.

\begin{description}

\item \textbf{Parameters:} User ID, Account e-mail address

\item \textbf{Precondition:} The user has no currently registered Paypal account connected to his account.

\item \textbf{Post-condition:} The account information is successfully added to
the database.

\item \textbf{Output messages:} The user receives a message saying that his
payment option has been registered into the system, and is asked if he wants to
make it the default payment option.

\item \textbf{Triggering:}
\begin{enumerate}
\item After entering data, press 'Add account' button.

\item The information is sent to the system server.

\item The system server sends the information to a Paypal server, and
confirms data.

\item If confirmation successful, the system server stores the data in the
database.

\item A message is sent to the client, confirming the addition of his payment
option.

\end{enumerate}

\end{description}

%% Paypal removing option
\subsection{Remove Paypal payment option}

 When editing his personal data, the client needs to be
 able to remove a Paypal account from his list of payment options. This
 operation is used to remove an account from the user's account in the system's
 database.

\begin{description}

\item \textbf{Parameters:} User ID, Account e-mail address

\item \textbf{Precondition:} The user has already registered a valid
Paypal account to his account.
\item \textbf{Precondition1 :} The user has at least one valid payment option
registered to his account, apart from the Paypal option.

\item \textbf{Post-condition:} The account information is successfully removed
from the database.

\item \textbf{Output messages:} The user receives a message saying that his
payment option has been removed from the system.

\item \textbf{Triggering:}
\begin{enumerate}
\item After selecting Paypal payment option, click on 'Remove'

\item A request is sent to the system server, asking to remove the Paypal account
from the client's account.

\item The system removes the Paypal account from the client's account and selects a
new default payment option.

\item A message is sent to the client, confirming the removal of his Paypal account from his account.

\end{enumerate}

\end{description}

%% American express payment option
\subsection{Add AmericanExpress payment}

 When registering and also editing his personal data, the client needs to be
 able to add a AmericanExpress card as a payment option. This operation is used
 in both instances, to handle the user's data and store it safely.

\begin{description}

\item \textbf{Parameters:} User ID, Card number, Card holder, Security number

\item \textbf{Precondition:} The user has no currently registered AmericanExpress
card connected to his account.

\item \textbf{Post-condition:} The card information is successfully added to the
database.

\item \textbf{Output messages:} The user receives a message saying that his
payment option has been registered into the system, and is asked if he wants to
make it the default payment option.

\item \textbf{Triggering:}
\begin{enumerate}
\item After entering data, press 'Add card' button.

\item The information is sent to the system server.

\item The system server sends the information to a AmericanExpress server, and
confirms data.

\item If confirmation successful, the system server stores the data in the
database.

\item A message is sent to the client, confirming the addition of his payment
option.

\end{enumerate}

\end{description}

%% American express removing option
\subsection{Remove AmericanExpress payment option}

 When editing his personal data, the client needs to be
 able to remove an AmericanExpress card from his list of payment options. This
 operation is used to remove a card from the user's account in the system's database.

\begin{description}

\item \textbf{Parameters:} User ID, Card number, Card holder, Security number

\item \textbf{Precondition:} The user has already registered a valid
AmericanExpress card to his account.
\item \textbf{Precondition1 :} The user has at least one valid payment option
registered to his account, apart from the AmericanExpress option.

\item \textbf{Post-condition:} The card information is successfully removed from
the database.

\item \textbf{Output messages:} The user receives a message saying that his
payment option has been removed from the system.

\item \textbf{Triggering:}
\begin{enumerate}
\item After selecting AmericanExpress payment option, click on 'Remove'

\item A request is sent to the system server, asking to remove the AmericanExpress card
from the client's account.

\item The system removes the AmericanExpress card from the client's account and selects a
new default payment option.

\item A message is sent to the client, confirming the removal of his AmericanExpress
card from his account.

\end{enumerate}

\end{description}