\section{Librarian}
\label{operation:Librarian}

This subsection provides a detailed description of librarian specific
functionalities.

\subsection{Search Customer}

The ''Search Customer'' functionality is only available for librarians. As the
title suggests, the librarian is able to look for customers using the
customers ID

\begin{description}

\item \textbf{Parameters:} Customer ID (provided by the customer), or other
identifable information

\item \textbf{Precondition:} The librarian must be logged into the system and
received the necessary information to look up the customer.

\item \textbf{Post-condition:} The librarian is now able to further inspect the
customer and take further actions

\item \textbf{Output messages:} Prints the results in a table containing First
name, Last name, internal Customer ID as well as account status (Active,
Overdue, Suspended).

\item \textbf{Triggering:}
\begin{enumerate}
\item Enter the customer ID or any identfiable information into the search bar.
\item Select ''Customer'' from the adjoining dropdown menu.
\item Hit the ''Search'' button and a list of all customers matching the query
will be displayed
\end{enumerate}

\end{description}

\subsection{Inspect Customer}

The ''Inspect'' button becomes available as soon as the librarian has searched
for a customer, so he can check the info available on a customer

\begin{description}

\item \textbf{Parameters:} None

\item \textbf{Precondition:} The librarian must be logged into the system,
received the necessary information to look up the customer and started a query
for said customer.

\item \textbf{Post-condition:} The librarian is now able to grant loans to the
inspected customers, initiate a book return or suspend his account.

\item \textbf{Output messages:} None.

\item \textbf{Triggering:} Press the ''Inspect'' button after searching for
customers and selecting the one you want to inspect.

\end{description}

\subsection{Search Book}

The ''Search Book'' functionality is available for the librarian. As the title
suggest the librarian is able to view the books in the catalogue as well as
perform other book related tasks.

\begin{description}

\item \textbf{Parameters:} Book name, ISBN, author, or nothing if he intends to
see the full catalogue

\item \textbf{Precondition:} The librarian must be logged into the system and
received the necessary information to look up the book (if applicable).

\item \textbf{Post-condition:} The librarian is now able to check the status of
all the books matching the search term, as well as add books to the catalogue,
or remove one from it.

\item \textbf{Output messages:} Prints the results in a table containing ISBN,
Author, full name, internal ID, as well as loan status (available, loaned,
overdue).

\item \textbf{Triggering:}
\begin{enumerate}
\item Enter the ISBN or any identfiable information into the search bar.
\item Select ''Book'' from the adjoining dropdown menu.
\item Hit the ''Search'' button and a list of all books matching the query
will be displayed
\end{enumerate}

\end{description}

\subsection{Add Book}

The ''Add Book'' functionality is available for the librarian. As the title
suggest the librarian is able to add books to the catalogue after opening the
latter through the ''Search Book'' functionality.

\begin{description}

\item \textbf{Parameters:} ISBN

\item \textbf{Precondition:} The librarian must be logged into the system,
opened the book catalogue and pressed ''Add book''.

\item \textbf{Post-condition:} The librarian is now able to add a book to the
catalogue by entering the books ISBN number into the popup windows and
confirming with the ''Add book'' button again. The system will automatically
gather the author and book name from a server and assign it a random internal
ID.

\item \textbf{Output messages:} None.

\item \textbf{Triggering:}
\begin{enumerate}
\item Open the library catalogue with the ''Search book'' functionality by
either searching for a specific book or viewing the whole catalogue by leaving
the search field blank.
\item Select ''Add book'' from under the table.
\item Enter the ISBN of the new book into the input field and confirm with ''Add
book'' to add it to the catalogue.
\end{enumerate}

\end{description}

\subsection{Remove Book}

The ''Remove Book'' functionality is available for the librarian. As the title
suggest the librarian is able to remove books from the catalogue after opening
the latter through the ''Search Book'' functionality.

\begin{description}

\item \textbf{Parameters:} None

\item \textbf{Precondition:} The librarian must be logged into the system,
opened the book catalogue.

\item \textbf{Post-condition:} The librarian is now able to remove a book from
the catalogue by selecting it in the catalogue and pressing ''Remove book''
under the catalogue display.

\item \textbf{Output messages:} None.

\item \textbf{Triggering:}
\begin{enumerate}
\item Open the library catalogue with the ''Search book'' functionality by
either searching for a specific book or viewing the whole catalogue by leaving
the search field blank.
\item Select a book from the catalogue.
\item Select ''Remove book'' from under the table.
\end{enumerate}

\end{description}
