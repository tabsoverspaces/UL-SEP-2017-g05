\section{Librarian}
\label{operation:Librarian}

This subsection provides a detailed description of librarian specific
functionalities.

\subsection{Access Customer Database}

The ''Access Customer Database'' functionality is only available for
librarians. As the title suggests, the librarian is able to access the database
with all the customers in it and search for a specific customer to perform
certain actions

\begin{description}

\item \textbf{Parameters:} None

\item \textbf{Precondition:} The librarian must be logged into the system.

\item \textbf{Post-condition:} The librarian is now able to look for a specific
customer using either their first name, last name, customer ID or account status
(active, overdue, suspended, banned) and inspect them to perform certain actions
on their account.

\item \textbf{Output messages:} None.

\item \textbf{Triggering:} Press the ''Customers'' button after logging into
your librarian account.

\end{description}

\subsection{Search Customer}

The ''Search Customer'' functionality is only available for librarians. As the
title suggests, the librarian is able to look for customers using either their
first name, last name, customer ID or account status (active, overdue, suspended, banned).

\begin{description}

\item \textbf{Parameters:} First name, last name, customer ID, account status or nothing if he intends to
see the full database. 

\item \textbf{Precondition:} The librarian must be logged into the system and
chosen to access the customer database.

\item \textbf{Post-condition:} The librarian is now able to further inspect the
customer and take further actions.

\item \textbf{Output messages:} Prints the results in a table containing First
name, Last name, customer ID as well as account status (active, overdue, suspended, banned).

\item \textbf{Triggering:}
\begin{enumerate}
\item Enter the first name, last name, customer ID or account status into the search bar.
\item Select the relevant property from the adjoining dropdown menu.
\item Hit the ''Search'' button and a list of all customers matching the query
will be displayed
\end{enumerate}

\end{description}

\subsection{Inspect Customer}

The ''Inspect'' button becomes available as soon as the librarian has searched
for a customer, so he can check the info available on a customer

\begin{description}

\item \textbf{Parameters:} None

\item \textbf{Precondition:} The librarian must be logged into the system,
received the necessary information to look up the customer and started a query
for said customer.

\item \textbf{Post-condition:} The librarian is now able to grant loans to the
inspected customers, initiate a book return, suspend and reinstate his account,
add notes and put in a request to either ban or unban his account.

\item \textbf{Output messages:} None.

\item \textbf{Triggering:} Press the ''Inspect'' button after searching for
customers and selecting the one you want to inspect.

\end{description}

\subsection{Add Loan}

The ''Add Loan'' functionality is available for the librarian. As the title
suggest the librarian is able to add a loan to a customers account after
inspecting the latter.

\begin{description}

\item \textbf{Parameters:} Book ID

\item \textbf{Precondition:} The librarian must be in the inspect window for a
customer and press the ''Add loan'' button.

\item \textbf{Post-condition:} The librarian has now added a loan to the
customers account with the loan starting date automatically being set and the
standard loan due date being 2 days from start.

\item \textbf{Output messages:} None.

\item \textbf{Triggering:}
\begin{enumerate}
\item Open the inspect window for the customer who wants to loan a book.
\item Select ''Add loan'' button from the bottom of the window.
\item Enter the book ID of the book the customer wants to loan.
\end{enumerate}

\end{description}

\subsection{Return Book}

The ''Return Book'' functionality is available for the librarian. As the title
suggest the librarian is able to let a customer return a book he loaned and add
it back to the pool of available books.

\begin{description}

\item \textbf{Parameters:} None

\item \textbf{Precondition:} The librarian must be in the inspect window for a
customer and have the currents loans being displayed, then choose the book to
return and click the ''Return book'' button.

\item \textbf{Post-condition:} The librarian has now successfully returned a
book from the customer to the book database and that book loan is added to the
loan history.

\item \textbf{Output messages:} None.

\item \textbf{Triggering:}
\begin{enumerate}
\item Open the inspect window for the customer who wants to loan a book.
\item Confirm you are shown the current loan and not the loan history.
\item Select the book to be returned.
\item Select ''Return book'' button from the bottom of the window.
\end{enumerate}

\end{description}

\subsection{Suspend Customer}

The ''Suspend Customer'' functionality is available for the librarian. As the
title suggest the librarian is able to suspend a customer disallowing him to
make further loans until his account is reinstated at a librarians discretion
(i.e. by paying a fine).

\begin{description}

\item \textbf{Parameters:} None

\item \textbf{Precondition:} The librarian must be in the inspect window for a
customer click the ''Suspend Customer'' button. Furthermore, this button won't
be displayed if the customer is already suspended, instead a ''Reinstate
Customer'' button will be shown.

\item \textbf{Post-condition:} The librarian has now suspended a customer and he
is unable to make further loans until he is reinstated.

\item \textbf{Output messages:} None.

\item \textbf{Triggering:}
\begin{enumerate}
\item Open the inspect window for the customer to be suspended.
\item Select ''Suspend Customer'' button from the bottom of the window.
\end{enumerate}

\end{description}

\subsection{Reinstate Customer}

The ''Reinstate Customer'' functionality is available for the librarian. As the
title suggest the librarian is able to reinstate a customer reallowing him to
make further loans.

\begin{description}

\item \textbf{Parameters:} None

\item \textbf{Precondition:} The librarian must be in the inspect window for a
customer click the ''Reinstate Customer'' button.

\item \textbf{Post-condition:} The librarian has now reinstated a customer and
he can once again request loans.

\item \textbf{Output messages:} None.

\item \textbf{Triggering:}
\begin{enumerate}
\item Open the inspect window for the customer to be reinstated.
\item Select ''Reinstate Customer'' button from the bottom of the window.
\end{enumerate}

\end{description}

\subsection{Request to ban customer}

The ''Ban Customer'' functionality is available for the librarian. With this
functionality, the librarian is able to put in a request to ban the currently
inspected customer by providing a small paragraph as to why the customer should
be banned.

\begin{description}

\item \textbf{Parameters:} None

\item \textbf{Precondition:} The librarian must be in the inspect window for a
customer click the ''Ban Customer'' button.

\item \textbf{Post-condition:} The librarian has now issued a request to ban the
customer in question.

\item \textbf{Output messages:} None.

\item \textbf{Triggering:}
\begin{enumerate}
\item Open the inspect window for the customer to be banned.
\item Select ''Ban Customer'' button from the bottom of the window.
\item A new window opens where the librarian can enter a reason as to why the
customer should be banned.
\item Finally the librarian clicks the ''Send'' which sends his request off for
review by a higher ranked person.
\end{enumerate}

\end{description}

\subsection{Request to unban customer}

The ''Unban Customer'' functionality is available for the librarian. With this
functionality, the librarian is able to put in a request to unban the currently
inspected customer by providing a small paragraph as to why the customer should
be unbanned.

\begin{description}

\item \textbf{Parameters:} None

\item \textbf{Precondition:} The librarian must be in the inspect window of a
currently banned customer and click the ''Unban Customer'' button.

\item \textbf{Post-condition:} The librarian has now issued a request to ban the
customer in question.

\item \textbf{Output messages:} None.

\item \textbf{Triggering:}
\begin{enumerate}
\item Open the inspect window for the customer to be unbanned.
\item Select ''Unban Customer'' button from the bottom of the window.
\item A new window opens where the librarian can enter a reason as to why the
customer should be unbanned.
\item Finally the librarian clicks the ''Send'' which sends his request off for
review by a higher ranked person.
\end{enumerate}

\end{description}

\subsection{Notes}

The ''Notes'' functionality is available for the librarian. With this
functionality, the librarian is able to add notes to a customer account like why
he has been suspended and what he has to do to be reinstated so the librarian
doing his reinstation knows why he was suspended in this example.

\begin{description}

\item \textbf{Parameters:} None

\item \textbf{Precondition:} The librarian must be in the inspect window of a
customer and click into the input field at the bottom to allow notes to be added
to his account.

\item \textbf{Post-condition:} The librarian has now added a note to the
customers account.

\item \textbf{Output messages:} None.

\item \textbf{Triggering:}
\begin{enumerate}
\item Open the inspect window for a customer.
\item Click into the input field at the bottom of the window.
\item The librarian can now add comments to the customers account.
\item When the librarian is done adding comments and notes to the customers
account, no further actions need to be taken.
\end{enumerate}

\end{description}

\subsection{Access Book Database}

The ''Access Books Database'' functionality is only available for
librarians. As the title suggests, the librarian is able to access the database
with all the customers in it and search for a specific customer to perform
certain actions

\begin{description}

\item \textbf{Parameters:} None

\item \textbf{Precondition:} The librarian must be logged into the system.

\item \textbf{Post-condition:} The librarian is now able to look for a specific
book using either the ISBN, the author or the title. Furthermore he is able to
add books to the database and remove some.

\item \textbf{Output messages:} None.

\item \textbf{Triggering:}Press the ''Books'' button after logging into
your librarian account.

\end{description}

\subsection{Search Book}

The ''Search Book'' functionality is available for the librarian. As the title
suggest the librarian is able to view the books in the database as well as
perform other book related tasks.

\begin{description}

\item \textbf{Parameters:} Book name, ISBN, author, or nothing if he intends to
see the full database.

\item \textbf{Precondition:} The librarian must be logged into the system,
chosen to access the book database and received the necessary information to
look up the book (if applicable).

\item \textbf{Post-condition:} The librarian is now able to check the status of
all the books matching the search term, as well as add books to the database,
or remove one from it.

\item \textbf{Output messages:} Prints the results in a table containing ISBN,
Author, full name, internal ID, as well as loan status (available, loaned,
overdue).

\item \textbf{Triggering:}
\begin{enumerate}
\item Enter the book name, ISBN, author into the search bar.
\item Select the relevant property from the adjoining dropdown menu.
\item Hit the ''Search'' button and a list of all books matching the query
will be displayed
\end{enumerate}

\end{description}

\subsection{Add Book}

The ''Add Book'' functionality is available for the librarian. As the title
suggest the librarian is able to add books to the database after opening the
latter through the ''Search Book'' functionality.

\begin{description}

\item \textbf{Parameters:} ISBN

\item \textbf{Precondition:} The librarian must be logged into the system,
opened the book database and pressed ''Add book''.

\item \textbf{Post-condition:} The librarian is now able to add a book to the
database by entering the books ISBN number into the popup windows and
confirming with the ''Add book'' button again. The system will automatically
gather the author and book name from a server and assign it a random internal
ID.

\item \textbf{Output messages:} None.

\item \textbf{Triggering:}
\begin{enumerate}
\item Open the library database with the ''Search book'' functionality by
either searching for a specific book or viewing the whole database by leaving
the search field blank.
\item Select ''Add book'' from under the table.
\item Enter the ISBN of the new book into the input field and confirm with ''Add
book'' to add it to the database.
\end{enumerate}

\end{description}

\subsection{Remove Book}

The ''Remove Book'' functionality is available for the librarian. As the title
suggest the librarian is able to remove books from the database after opening
the latter through the ''Search Book'' functionality.

\begin{description}

\item \textbf{Parameters:} None

\item \textbf{Precondition:} The librarian must be logged into the system,
opened the book database.

\item \textbf{Post-condition:} The librarian is now able to remove a book from
the database by selecting it in the database and pressing ''Remove book''
under the database display.

\item \textbf{Output messages:} None.

\item \textbf{Triggering:}
\begin{enumerate}
\item Open the library database with the ''Search book'' functionality by
either searching for a specific book or viewing the whole database by leaving
the search field blank.
\item Select a book from the database.
\item Select ''Remove book'' from under the table.
\end{enumerate}

\end{description}

\subsection{Check loan\&reservations requests}

The ''Check loan\&reservations request'' functionality is available for the
librarian.
As the title suggest the librarian is able to check loans and reservations made
via the mobile and either grant or deny them.

\begin{description}

\item \textbf{Parameters:} None.

\item \textbf{Precondition:} The librarian must be logged into the system, be in
any database or the homescreen and press ''Loan\&Reservation requests'' button
in the top right corner.

\item \textbf{Post-condition:} The librarian is now able to see any loan or
reservation requests made via the mobile app and decide which to grant and which
not.

\item \textbf{Output messages:} None.

\item \textbf{Triggering:}
\begin{enumerate}
\item Open any library database and click the ''Loan\&Reservation requests''
button on the top right.
\end{enumerate}

\end{description}

\subsection{Grant Loan/Reservation request}

The ''Grant Loan/Reservation request'' functionality is available for the
librarian.
As the title suggest the librarian is able to grant loans or reservations
requested via the mobile app.

\begin{description}

\item \textbf{Parameters:} None.

\item \textbf{Precondition:} The librarian must be logged into the system, be in
any database or the homescreen, have pressed ''Loan\&Reservation requests''
button in the top right corner and is now in the ''Loan\&Reservation requests''
window.

\item \textbf{Post-condition:} The librarian has now successfully granted a
Loan/Reservation request

\item \textbf{Output messages:} None.

\item \textbf{Triggering:}
\begin{enumerate}
\item Open any library database and click the ''Loan\&Reservation requests''
button on the top right.
\item Select one of the requests
\item Click ''Grant Loan/Reservation''
\end{enumerate}

\end{description}

\subsection{Deny Loan/Reservation request}

The ''Deny Loan/Reservation request'' functionality is available for the
librarian.
As the title suggest the librarian is able to deny loans or reservations
requested via the mobile app.

\begin{description}

\item \textbf{Parameters:} None.

\item \textbf{Precondition:} The librarian must be logged into the system, be in
any database or the homescreen, have pressed ''Loan\&Reservation requests''
button in the top right corner and is now in the ''Loan\&Reservation requests''
window.

\item \textbf{Post-condition:} The librarian has now successfully denied a
Loan/Reservation request

\item \textbf{Output messages:} None.

\item \textbf{Triggering:}
\begin{enumerate}
\item Open any library database and click the ''Loan\&Reservation requests''
button on the top right.
\item Select one of the requests
\item Click ''Deny Loan/Reservation''
\end{enumerate}

\end{description}

\subsection{Check messages}

The ''Check messages'' functionality is available for the librarian.
As the title suggest the librarian is able to check his messages send to the
system administrator and those received by him.

\begin{description}

\item \textbf{Parameters:} None.

\item \textbf{Precondition:} The librarian must be logged into the system, be in
any database or the homescreen and press ''Messages'' button in the
top right corner.

\item \textbf{Post-condition:} The librarian is now able to see any messages
exchanged between him and the system administrator as well as send him new
messages.

\item \textbf{Output messages:} None.

\item \textbf{Triggering:}
\begin{enumerate}
\item Open any library database and click the ''Messages''
button on the top right.
\end{enumerate}

\end{description}

\subsection{Send messages}

The ''Send messages'' functionality is available for the librarian.
As the title suggest the librarian is able to send messages to the
system administrator.

\begin{description}

\item \textbf{Parameters:} None.

\item \textbf{Precondition:} The librarian must be logged into the system, be in
any database or the homescreen, have pressed ''Messages'' button in the
top right corner and is now in the messaging window.

\item \textbf{Post-condition:} The librarian has now successfully send a message
to the system administrators.

\item \textbf{Output messages:} None.

\item \textbf{Triggering:}
\begin{enumerate}
\item Open any library database and click the ''Messages''
button on the top right.
\item Click in the textfield below the messages
\item Enter the message to be send into the field
\item Press the button to the right labeled ''Send''
\end{enumerate}

\end{description}	
