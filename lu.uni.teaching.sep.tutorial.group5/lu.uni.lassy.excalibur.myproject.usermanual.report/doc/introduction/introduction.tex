\chapter{Introduction}
\label{chap:introduction}

\section{Scope}

This document provides information regarding the usage of the \mysystemname
software.
 
This document is not intended to provide information about how to
 connect, deploy, configure, or use any external device or
 third-party software system that is rqeuired for the correct functioning of
 \mysystemname.

This document may be used with other documents provided by third-party
 companies to have an overall view and correct understanding of the environment
 and procedures where the software system \mysystemname is aimed to be deployed
 and run.


\section{Purpose}

This user manual is aimed towards three user groups the system administrator,
librarians and clients. It is aimed at the System Administrator, who will be using the desktop app of the sysadmin to access all its related functionality. 

This document's purpose is to provide a way to help the system administrators in
understanding, navigating the different panels and explaining how the different functionalities work and are used. 

This document is intended to be read by system administrators making it easier
for them to adapt and understand the application.

This user manual is also aimed towards the library's clients, who will be using
the mobile app in order to gain access to their library account.

The purpose of this document is to help the users in navigating the different
panels, as well as teach them the functionality that the application provides.

This document is meant to be read by the clients in order to make it more easy
for them to adapt to using the application.





\section{Intended audience}
This document is meant to be used by all users (System Administrator, Librarians
and Clients) of the \mysystemname software. The users must be in a library
environment (School library or public library) in order for these applications
to work. The intended System Administrator is someone with a good tech
background who can easily solve technological issues for the librarians after
using the messages system. 


\section{\mysystemname}
This software is designed to be used in accordance with a library system.
The users are able to browse through list of loan-available books, send loan
request for a certain book(s) and send reservation requests for temporarily
unavailable books. 


The system administrators are able to browse through the list of librarians,
add, edit and delete librarians and send and receive messages to librarians. The
system administrator is also able to view the librarians history through a log.


\subsection{Actors \& Functionalities}
Overview of all the \textbf{\emph{\glspl{actor}}} interacting with the software
being them either humans (called end-users in the standard
\cite{IEEE-2001-userdocumentation}) or not. For each actor, describe the main
software functions that are offered to him. Structure of this sub-section MUST
be by actor/functionalities.


\subsection{Operating environment}
The software will be deployed on both desktops and mobile platforms. The desktop
version will be supported in the three most popular systems which are Windows,
macOS and Linux. As for the mobile platforms there will be both an Android
version as well as iOS version.

\section{Document structure}

The document is structured where in each chapter you will see information about
the system administrators before librarians and then clients. 







