\chapter{Software operations}
\label{chap:soptware_operations}


Explain each allowed software operations (i.e. an atomic unit of treatment, a service, a functionality) including a brief description of the operation, required parameters, optional parameters, default options, required steps to trigger the operation, assumptions upon request of the operation and expected results of executing such operation.
Describe how to recognise that the operation has successfully been executed or
abnormally terminated. The template given below (i.e. section \ref{operation:MyOperation} has to be used).

Group the operations devoted to the needs of specific actors. Common
operations to several actors may be grouped and presented once to avoid redundancy.


\section{MyOperation}
\label{operation:MyOperation}
The system operator creates and adds a new crisis to the system after being
informed by a third party (citizen, organization) and selects a crisis handler for the crisis.

\begin{description}

\item \textbf{Parameters:} Reporter Personal Information, Crisis Information, Crisis Handler
\item \textbf{Precondition:} The system operator is logged in and has received information from a reporter.
\item \textbf{Post-condition:} A new crisis has been added to the system and the new crisis has been assigned to a crisis handler, the Handler has received an automatic notification from the system.
\item \textbf{Output messages:} The selected Crisis Handler will be notified
automatically once the crisis has been created.

\item \textbf{Triggering:}
\begin{enumerate}
\item From within the crisis management window fill out the required entries related to the personal information of the reporter such as name and phone number.
\item Fill out the entries related to the crisis type, impacted area, priority, description, GPS coordinates, address and finally choose a Crisis Handler from the combo box.
\item Click on the “Submit” button in and add the entry to the database.
\end{enumerate}

 
\end{description}

 
\subsection{MyExample1}
Examples should illustrate the use of \textbf{complex operations}.

Each example must show how the actor uses the software operation under
description to achieve (at least one of) its expected outcome.

It might be required to include GUI screenshots to illustrate the example.



