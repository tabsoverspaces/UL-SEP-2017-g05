\section{System Administrator}
\label{operation:System Administrator}

This subsection provides a detailed description of system administrator specific functionalities.

\subsection{Add Librarian}

The add librarian functionality is only available for the system administrator. As the title suggests, the system administrator creates and adds a new librarian account being informed from organization/library. 

\begin{description}

\item \textbf{Parameters:} ID (Randomly generated), First Name, Last Name, Date of Birth

\item \textbf{Precondition:} The system administrator must be logged into the system and must have received the information about the librarian too add. 

\item \textbf{Post-condition:} The new librarian account been added to the database system and the librarian and the organization have received an automated notification so that the account can be finalized. 

\item \textbf{Output messages:} The librarian and the organization will receive an automated notification informing that the account can now be finalized. 

\item \textbf{Triggering:}
\begin{enumerate}

\item From the users main window, click on the Add Librarian button to open the add librarian page. 

\item From the users main window, click on the Add Librarians button to open the add librarian page. 

\item Once the add librarian page opens, fill out all the required parameters with information related to the new librarian that is being created. 

\item Click on the Submit button to add the new librarians information to a database. 

\end{enumerate}

\end{description}

\subsection{Edit Librarian}

The edit librarian functionality is used when editing user information. This includes but is not limited to the first name, last name and date of birth as well as password eventually. The goal of this feature is to allow the system administrator to edit a librarians information at the request of the librarian. 

\begin{description}

\item \textbf{Parameters:} ID, First Name, Last Name

\item \textbf{Precondition:} The system administrator must be logged in, at the users page and must have the different information of the librarian that needs to be edited.

\item \textbf{Post-condition:} The edits for the librarian have been saved and a system popup appears starting that the librarian information has successfully been edited and saved. 

\item \textbf{Output messages:} The system administrator will see a system popup informing the system administrator that the librarian information has successfully been edited and saved.

\item \textbf{Triggering:}
\begin{enumerate}

\item Click on the "Edit Librarian" button that is found in the users main page.

\item Once the edit librarian page has opened, either find the user manually or search the user via the search field at the top.

\item Click on the Librarian ID to open up the edit details page. 

\item Edit the necessary details of the user that need to be edited. Click submit when complete. You will be automatically be taken back to the users page.

\end{enumerate}

\end{description}

\subsection{Delete Librarian}

The delete librarian is once again only available for the system administrator. The delete librarian functionality is used when deleting an existing librarian account that will no longer be used due to different reasons such as; the librarian no longer works, an unfortunate event has happened to the librarian.

\begin{description}

\item \textbf{Parameters:} ID (Randomly generated), First Name, Last Name

\item \textbf{Precondition:} The system administrator must be logged into the system and must have received the information about the librarian too remove. 

\item \textbf{Post-condition:} The librarian account has been
successfully deleted from the database system and the sysadmin will receive an
onscreen notification informing that the user has been successfully deleted.

\item \textbf{Output messages:} The librarian and the organization will receive an automated notification informing that the account can now be finalized. 

\item \textbf{Triggering:}
\begin{enumerate}

\item Click on the Delete Librarian button at the bottom of the users main

\item Click on the Delete Librarians button at the bottom of the users main
window to open the users deletion page.

\item Once in the users deletion page, find the user that needs to be deleted manually or via the search field.

\item Once the user has been found, click the red x to remove the user permanently from the database system. 

\item Click on the "back/done" button to return to users page.

\end{enumerate}

\end{description}

\subsection{Messages}

The messages functionality is a system which allows Librarians to contact System
Administrators and vice versa. Its main purpose is to allow an easy communication system to help System Administrators and Librarians communicate issues and changes more easily.

\subsubsection{New Message}

The new message functionality like the title suggests allows a system administrator to create a new message to send to a librarian. 

\begin{description}

\item \textbf{Parameters:} Librarian ID, First Name, Last Name, Message

\item \textbf{Precondition:} The system administrator must be logged in and must have received the information (ID, First Name, Last Name) of the librarian to whom the message has to be sent.

\item \textbf{Post-condition:} The message is sent to the Librarian and the
system administrator will see the message as a rectangle in the chat.

\item \textbf{Output messages:} None.

\item \textbf{Triggering:}
\begin{enumerate}

\item Click on the new message button at the bottom left of the messages main panel. 

\item Select the Librarian that needs to receive the message from the drop down menu.

\item Type the message you want to send in the text box at the bottom. 

\item Click the send button at the top left when complete.  

\end{enumerate}

\end{description}